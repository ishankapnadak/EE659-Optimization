\documentclass[12pt]{article}
\usepackage{amsmath, amssymb, amsfonts, amsthm, mathtools,mathrsfs}
\usepackage{bm}
\usepackage{thmtools}
\usepackage[utf8]{inputenc}
\usepackage[inline]{enumitem}
\usepackage[colorlinks=true]{hyperref}
\usepackage{multicol}
\usepackage{witharrows}
\usepackage{tikz}
\usetikzlibrary{automata,positioning}
\usetikzlibrary{decorations.markings}
\usepackage{verbatim}
\usetikzlibrary{arrows.meta} 
\usepackage{witharrows}
\usepackage[useregional, showdow]{datetime2}
\usepackage{physics}
\DTMlangsetup[en-GB]{abbr}
\usepackage{xcolor}
\usepackage[normalem]{ulem}
\usetikzlibrary{chains,shapes.multipart}
\usepackage{algorithm}
\usepackage{algpseudocode}

\usepackage{bbm}
\usepackage[page,toc,titletoc,title]{appendix}
\usepackage{tocloft}
\setlength\parindent{0pt}
\usepackage{parskip}

\def\D{\mathrm{d}}
\def\I{\mathbb{I}}
\def\P{\mathbb{P}}
\def\E{\mathbb{E}}
\def\Var{\text{Var}}
\def\F{\mathcal{F}} 
\def\ccdf{\overline{F}} 
\newcommand*{\thead}[1]{\multicolumn{1}{c}{\bfseries #1}}
\renewcommand{\arraystretch}{1}
\newcommand{\indep}{\perp \!\!\! \perp}

\usepackage[framemethod=tikz]{mdframed}
\mdfdefinestyle{theoremstyle}{%
	% linecolor=gray,linewidth=1pt,%
	% frametitlerule=true,%
	frametitlebackgroundcolor=white,
	% backgroundcolor=  gray!20,	
	bottomline=false, topline=false, rightline=false, leftline=true,
	innerlinewidth=0.7pt, outerlinewidth=0.2pt, middlelinewidth=2pt, middlelinecolor=white, %
	innerleftmargin=6pt,
	% innertopmargin=-1pt,
	skipabove=10pt,
	% fontcolor=blue,
	% innerbottommargin=-0.5pt,
}
\mdtheorem[style=theoremstyle]{defn}[thm]{Definition}[section]
\mdtheorem[style=theoremstyle]{lem}[thm]{Lemma}
\mdtheorem[style=theoremstyle]{prop}[thm]{Proposition}
\mdtheorem[style=theoremstyle]{thm}{Theorem}[section]
\mdtheorem[style=theoremstyle]{cor}{Corollary}[section]


\newcommand*{\doublerule}{\hrule width \hsize height 1pt \kern 0.5mm \hrule width \hsize height 2pt}
\newcommand{\doublerulefill}{\leavevmode\leaders\vbox{\hrule width .1pt\kern1pt\hrule}\hfill\kern0pt}
\def\ddfrac#1#2{\displaystyle\frac{\displaystyle #1}{\displaystyle #2}}
\newcommand{\interior}[1]{%
  {\kern0pt#1}^{\mathrm{o}}%
}

%\newcommand{\Res}{\operatorname{Res}}

\theoremstyle{definition}
% \numberwithin{thm}{section}
% \newtheorem{lem}[thm]{Lemma}
% \newtheorem{defn}[thm]{Definition}
% \newtheorem{prop}[thm]{Proposition}
% \newtheorem{cor}[thm]{Corollary}
% \newtheorem{ex}{Example}


\let\emptyset\varnothing

\usepackage{titlesec}
\titleformat{\section}[block]{\Large\filcenter\bfseries}{\S\thesection.}{0.25cm}{\Large}
\titleformat{\subsection}[block]{\large\bfseries\sffamily}{\S\S\thesubsection.}{0.2cm}{\large}

\usepackage[a4paper]{geometry}
\usepackage{lipsum}
\usepackage{xcolor,cancel}

\usepackage{cleveref}
\crefname{thm}{Theorem}{Theorems}
\crefname{lem}{Lemma}{Lemmas}
\crefname{defn}{Definition}{Definitions}
\crefname{prop}{Proposition}{Propositions}
\crefname{cor}{Corollary}{Corollaries}
\crefname{equation}{}{}
\DeclareMathOperator*{\argmin}{arg\,min}
\DeclareMathOperator*{\Vol}{Vol}
\DeclareMathOperator*{\diag}{diag}

\usepackage{mdframed}
\newenvironment{blockquote}
{\begin{mdframed}[skipabove=0pt, skipbelow=0pt, innertopmargin=4pt, innerbottommargin=4pt, bottomline=false,topline=false,rightline=false, linewidth=2pt]}
{\end{mdframed}}
\newenvironment{soln}{\begin{proof}[Solution]}{\end{proof}}

\title{A First Course in Optimization\\}
\author{Ishan Kapnadak}
\date{Autumn Semester 2022-23\\~\\Updated on: \textcolor{blue}{\DTMToday}}

\begin{document}
\tikzset{lab dis/.store in=\LabDis,
  lab dis=-0.4,
  ->-/.style args={at #1 with label #2}{decoration={
    markings,
    mark=at position #1 with {\arrow{>}; \node at (0,\LabDis) {#2};}},postaction={decorate}},
  -<-/.style args={at #1 with label #2}{decoration={
    markings,
    mark=at position #1 with {\arrow{<}; \node at (0,\LabDis)
    {#2};}},postaction={decorate}},
  -*-/.style args={at #1 with label #2}{decoration={
    markings,
    mark=at position #1 with {{\fill (0,0) circle (1.5pt);} \node at (0,\LabDis)
    {#2};}},postaction={decorate}},
  }
\maketitle

\begin{abstract}
    \begin{center}
        Lecture Notes for the course EE 659: A First Course in Optimization taught in Spring 2022 by Prof. Vivek Borkar. Additional references include \textit{A first course in optimization} by Rangarajan K. Sundaram, \textit{Optimization by vector space methods} by David Luenberger, and \textit{Nonlinear programming} by Dimitri P. Bertsekas
    \end{center}
\end{abstract}

\tableofcontents
\newpage

\section{Lecture 1}
\medskip

\begin{defn}[Open Ball]
    The \emph{open ball} of radius $\epsilon$ centered around $\mathbf{x}_0 \in \mathbb{R}^d$ is defined as
    \[
        B_{\epsilon}(\mathbf{x}_0) \vcentcolon= \left\{ \mathbf{x} \in \mathbb{R}^d \colon \norm{\mathbf{x} - \mathbf{x}_0} < \epsilon \right\}.
    \]
\end{defn}
\begin{defn}[Closed Ball]
    The \emph{closed ball} of radius $\epsilon$ centered around $\mathbf{x}_0 \in \mathbb{R}^d$ is defined as
    \[
        \overline{B}_{\epsilon}(\mathbf{x}_0) \vcentcolon= \left\{ \mathbf{x} \in \mathbb{R}^d \colon \norm{\mathbf{x} - \mathbf{x}_0} \leq \epsilon \right\}.
    \]
\end{defn}
\begin{defn}[Open and Closed Sets]
    A set $A \subset \mathbb{R}^d$ is said to be \emph{open} if for all $\mathbf{x} 
    \in A$, there exists an $\epsilon > 0$ such that $B_{\epsilon}(\mathbf{x}) \subset A$. A set $A$ is said to be \emph{closed} if $A^{\mathsf{c}}$ is open.
\end{defn}

We have the following properties for open and closed sets.

\begin{enumerate}
    \item Let $\mathcal{I}$ be an arbitrary index set. If $A_{\alpha}$ is open for each $\alpha \in \mathcal{I}$, then 
    \[
        \bigcup_{\alpha \in \mathcal{I}} \, A_{\alpha}
    \]
    is open. In other words, open sets are closed under arbitrary unions.

    \item Let $\mathcal{I}$ be a finite index set. If $A_{\alpha}$ is open for each $\alpha \in \mathcal{I}$, then 
    \[
        \bigcap_{\alpha \in \mathcal{I}} \, A_{\alpha}
    \]
    is open. In other words, open sets are closed under finte intersections.

    \item Let $\mathcal{I}$ be an arbitrary index set. If $A_{\alpha}$ is closed for each $\alpha \in \mathcal{I}$, then 
    \[
        \bigcap_{\alpha \in \mathcal{I}} \, A_{\alpha}
    \]
    is closed. In other words, closed sets are closed under arbitrary intersections.

    \item Let $\mathcal{I}$ be a finite index set. If $A_{\alpha}$ is closed for each $\alpha \in \mathcal{I}$, then 
    \[
        \bigcup_{\alpha \in \mathcal{I}} \, A_{\alpha}
    \]
    is closed. In other words, closed sets are closed under finite unions.
\end{enumerate}

\begin{defn}[Convergence of a sequence]
    Let $\left\{ \mathbf{x}_n \right\}$ be a sequence in $\mathbb{R}^d$. Then, $\left\{\mathbf{x}_n\right\}$ converges to $\mathbf{x}^*$ (written $\mathbf{x}_n \to \mathbf{x}^*$) if for all $\epsilon > 0$, there exists an $n_0 \in \mathbb{N}$ such that
    \[
        \mathbf{x}_n \in B_{\epsilon}(\mathbf{x}^*) \quad \forall n > n_0.
    \]
    Equivalently, $\norm{\mathbf{x}_n - \mathbf{x}^*} \to 0$.
\end{defn}

\begin{defn}[Closure]
    Let $A \subset \mathbb{R}^d$. The \emph{closure} of $A$ (denoted $\overline{A}$) is the smallest closed set containing $A$. Equivalently, $\overline{A}$ is the intersection of all closed sets containing $A$.
\end{defn}

\begin{defn}[Interior]
    Let $A \subset \mathbb{R}^d$. The \emph{interior} of $A$ (denoted $\interior{A}$) is the largest open set contained in $A$. Equivalently, $\interior{A}$ is the union of all open sets contained in $A$.
\end{defn}
Note that by definition, we have $\interior{A} \subset A \subset \overline{A}$.

\begin{defn}[Boundary]
    Let $A \subset \mathbb{R}^d$. The \emph{boundary} of $A$ is defined as
    \[
        \partial A \vcentcolon= \overline{A} \setminus \interior{A}.
    \]
\end{defn}
Note that for a closed set, $\overline{A} = A$, and for an open set, $\interior{A} = A$. 

\begin{prop}
    A set $A \subset \mathbb{R}^d$ is closed if and only if
    \[
        \mathbf{x}_n  \to \mathbf{x}^*, \mathbf{x}_n \in A \forall n \implies \mathbf{x}^* \in A.
    \]
\end{prop}
\begin{proof}
    Let $A \subset \mathbb{R}^d$ be closed, $\mathbf{x}_n \in A$ for all $n$, and $\mathbf{x}_n \to \mathbf{x}^*$. Assume to the contrary that $\mathbf{x}^* \notin A$. Then, $\mathbf{x}^* \in A^{\mathsf{c}}$, which is open by assumption. Thus, $\exists \epsilon > 0$ such that $B_{\epsilon}(\mathbf{x}^*) \subset A^{\mathsf{c}}$. This implies that $\mathbf{x}_n \notin B_{\epsilon}(\mathbf{x}^*)$ for all $n$, and thus $\mathbf{x}_n \not\to \mathbf{x}^*$, a contradiction. To prove the converse, assume that $A$ is not closed. Thus, there exists a $\Tilde{\mathbf{x}} \in \partial A$ such that $\Tilde{\mathbf{x}} \notin A$. Then, for all $\epsilon > 0$, $B_{\epsilon}(\Tilde{\mathbf{x}}) \cap A \neq \emptyset$. Let $\epsilon_n \downarrow 0$ and let $\mathbf{x}_n \in B_{\epsilon_n}(\Tilde{\mathbf{x}}) \cap A$. Then, $\mathbf{x}_n \to \Tilde{\mathbf{x}} \notin A$, a contradiction.
\end{proof}

\begin{defn}[Limit Point]
    Let $\left\{ \mathbf{x}_n \right\}$ be a sequence in $\mathbb{R}^d$. $\Tilde{\mathbf{x}}$ is a \emph{limit point} of $\left\{ \mathbf{x}_n \right\}$ if there exists a subsequence $\left\{ \mathbf{x}_{n_k} \right\}$ such that $\mathbf{x}_{n_k} \to \Tilde{\mathbf{x}}$.
\end{defn}

\begin{prop}
    $\left\{ \mathbf{x}_n \right\}$ converges if and only if $\left\{ \mathbf{x}_n \right\}$ has a unique limit point.    
\end{prop}

\begin{defn}[Supremum and Infimum]
    Let $A \subset \mathbb{R}$ be bounded. Then, 
    \begin{align*}
        \sup A &\vcentcolon= \text{ smallest } x \in \mathbb{R} \cup \{+\infty\} \text{ such that } y \in A \implies y \leq x, \\
        \inf A &\vcentcolon= \text{ largest } x \in \mathbb{R} \cup \{-\infty\} \text{ such that } y \in A \implies y \geq x.
    \end{align*}
\end{defn}

\begin{defn}[Cauchy Sequence]
    A sequence $\left\{ \mathbf{x}_n \right\}$ is said to be \emph{Cauchy} if $\displaystyle\lim_{m,n \uparrow \infty} \norm{\mathbf{x}_m - \mathbf{x}_n} = 0$.
\end{defn}

\begin{prop}
    Cauchy sequences are bounded.
\end{prop}
\begin{proof}
    Let $\left\{ \mathbf{x}_n \right\}$ be a Cauchy sequene and let $\epsilon > 0$. Pick $N$ large enough such that 
    \[
        n,m > N \implies \norm{\mathbf{x}_m - \mathbf{x}_n} < \epsilon.
    \]
    We then have
    \begin{align*}
        &\mathbf{x}_n \in B_{\epsilon}(\mathbf{x}_m) \quad \forall n > N \\
        &\implies \left\{ \mathbf{x}_n \colon n > N\right\} \text{ is bounded} \\
        &\implies \left\{ \mathbf{x}_n \right\} \text{ is bounded.} \qedhere
    \end{align*}
\end{proof} 
\begin{prop}
    Cauchy sequences have at most one limit point.
\end{prop}
\begin{proof}
    Suppose $\left\{ \mathbf{x}_n \right\}$ is a Cauchy sequence having two limit points, $\Tilde{\mathbf{x}}$ and $\overline{\mathbf{x}}$. Then, there exist subsequences $\mathbf{x}_{\Tilde{n}_k} \to \Tilde{\mathbf{x}}$ and $\mathbf{x}_{\overline{n}_l} \to \overline{\mathbf{x}}$. We then have
    \[
        \lim_{\Tilde{n}_k, \overline{n}_l \uparrow \infty} \norm{\mathbf{x}_{\Tilde{n}_k} - \mathbf{x}_{\overline{n}_l}} = 0 \implies \Tilde{\mathbf{x}} = \overline{\mathbf{x}}. \qedhere
    \]
\end{proof}
\begin{defn}[Complete Space]
    A metric space is \emph{complete} if every Cauchy sequence converges.
\end{defn}
\section{Lecture 2}

\begin{thm}[Bolzano-Weierstrass Theorem]
    Every bounded sequence in $\mathbb{R}^d$ has a convergent subsequence.
\end{thm}
\begin{proof}
    Suppose $d = 1$. Since $\left\{ x_n \right\}$ is bounded, we have $x_n \in [a,b]$ for all $n$ where $a,b \in \mathbb{R}$ and $a < b$. The idea is to keep halving the interval and pick a half interval containing infinitely many points. For example, consider the two half intervals $[a, \frac{a+b}{2}]$ and $[\frac{a+b}{2}, b]$. Since $\left\{ x_n \right\}$ has infinitely many points, at least one of these two half intervals has infinitely many points. Call this half interval $[a_1, b_1]$ and repeat this argument again for $[a_1, b_1]$. This gives us a sequence $\left\{ (a_n, b_n) \right\}$ satisfying 
    \begin{align*}
        a_0 \leq a_1 \leq a_2 \leq \cdots b &\implies a_n \to a^* \\
        b_0 \geq b_1 \geq b_2 \geq \cdots a &\implies b_n \to b^* \\
    \end{align*}
    where we define $a_0 \vcentcolon= a$ and $b_0 \vcentcolon= b$. Moreover, we have
    \[
        \abs{b_n - a_n} = \frac{b-a}{2^n} \to 0 \implies a^* = b^*.
    \]
    Since there are infinitely many $x_n$'s in $[a_k, b_k]$ for any $k$, pick $\Tilde{x}_k \in [a_k,b_k] \cap \{x_n\}$ such that $\Tilde{x}_k \neq \Tilde{x}_j$ for $j < k$. Thus, $\Tilde{x}_k \to a^* = b^*$. This can be generalised to $d > 1$ via induction and we leave this as an exercise to the reader.
\end{proof}

Note that the above argument does not generalise to infinite dimensions. For example, consider the complete orthonormal space
\[
    \mathcal{L}_2[0,T] \vcentcolon= \left\{ f \colon [0,T] \to \mathbb{R} \colon \int_0^T f^2(t) \, \D t < \infty \right\}
\]
with inner product
\[
    \langle f,g \rangle \vcentcolon= \int_0^T f(t) g(t) \, \D t.
\]
Consider an orthonormal basis $\{e_n\}$ satisfying 
\[
    \langle e_n, e_m \rangle = 
    \begin{cases}
        1 & \text{ if } n = m, \\
        0 & \text{ if } n \neq m.
    \end{cases}
\]
Note that $\norm{e_n - e_m} = \sqrt{2}$ whenever $n \neq m$ and thus $\{e_n\}$ has no convergent subsequence. 

\begin{prop}
    Let $f \colon C \subset \mathbb{R}^d \to \mathbb{R}$ be bounded from below. Let $\beta = \inf_{\mathbf{x} \in C} f(\mathbf{x})$. Then, $\exists \left\{ \mathbf{x}_n \right\} \in C$ such that $f(\mathbf{x}_n) \downarrow \beta$.  
\end{prop}

\begin{thm}[Weierstrass Theorem]
    Let $C \subset \mathbb{R}^d$ be closed and bounded, and let $f \colon C \to \mathbb{R}$ be continuous. Then, $f$ attains its minimum and maximum. 
\end{thm}
\begin{proof}
    Let $\left\{ \mathbf{x}_n \right\rangle \in C$ be such that $f(\mathbf{x}_n) \downarrow \beta \vcentcolon= \inf_{\mathbf{x} \in C} f(\mathbf{x})$. By Bolzano-Weierstrass, $\exists \left\{ \mathbf{x}_{n_k} \right\}$ such that $\mathbf{x}_{n_k} \to \mathbf{x}^*$. Since $C$ is closed, $\mathbf{x}^* \in C$. Since $f$ is continuous, $f(\mathbf{x}_{n_k}) \to f(\mathbf{x}^*) \implies f(\mathbf{x}^*) = \beta$. A similar argument holds for maximum.
\end{proof}

\begin{cor}
    Let $C \subset \mathbb{R}^d$ be closed and let $f \colon C \to \mathbb{R}$ be continuous and satisfy 
    \[
        \lim_{\norm{\mathbf{x}} \uparrow \infty} f(\mathbf{x}) = \infty.
    \]
    Then, $f$ attains its minimum on $C$. 
\end{cor}
\begin{proof}
    Let $\left\{ \mathbf{x}_n \right\}$ be such that $f(\mathbf{x}_n) \downarrow \beta \vcentcolon= \inf_{\mathbf{x} \in C} f(\mathbf{x})$. Then, $\left\{ \mathbf{x}_n \right\}$ is bounded, since otherwise $\exists \left\{ \mathbf{x}_{n_k} \right\}$ such that $\norm{\mathbf{x}_{n_k}} \uparrow \infty \implies f(\mathbf{x}_{n_k}) \to \infty \neq \beta$. The previous argument now follows through.
\end{proof}

\begin{defn}[Limit Supremum and Limit Infimum]
    Let $\left\{ x_n \right\} \in \mathbb{R}$. We define
    \begin{align*}
        \limsup_{n \uparrow \infty} x_n &\vcentcolon= \lim_{n \uparrow \infty} \, \sup_{m \geq n} x_m = \inf_{n \geq 1} \, \sup_{m \geq n} x_m \\
        \liminf_{n \uparrow \infty} x_n &\vcentcolon= \lim_{n \uparrow \infty} \, \inf_{m \geq n} x_m = \sup_{n \geq 1} \, \inf_{m \geq n} x_m
    \end{align*}
    We sometimes also denote the limit supremum as $\overline{\lim} x_n$ and the limit infimum as $\underline{\lim} x_n$.
\end{defn}

Note that $\limsup$ and $\liminf$ are always well-defined if we allow $\{\pm\infty\}$ as possibilities. This is because $\sup_{m \geq n} x_m$ is a non-increasing sequence and thus must converge (possibly to $-\infty$). Similarly, $\inf_{m \geq n} x_m$ is a non-decreasing sequence and thus must converge (possibly to $+\infty$). We also note that

\begin{enumerate}
    \item $\displaystyle \limsup_{n \uparrow \infty} x_n \geq \liminf_{n \uparrow \infty} x_n$.

    \item If $\displaystyle \limsup_{n \uparrow \infty} x_n = \liminf_{n \uparrow \infty} x_n = x^*$, then $x_n \to x^*$. 
\end{enumerate}

\begin{defn}[Lower and Upper Semicontinuous]
    $f \colon C \subset \mathbb{R}^d \to \mathbb{R}$ is said to be lower semicontinuous (l.s.c) if whenever $\mathbf{x}_n \to \mathbf{x}^*$ in $C$, then $\displaystyle \liminf_{n \uparrow \infty} f(\mathbf{x}_n) \geq f(\mathbf{x}^*)$. 

    $f \colon C \subset \mathbb{R}^d \to \mathbb{R}$ is said to be upper semicontinuous (u.s.c) if whenever $\mathbf{x}_n \to \mathbf{x}^*$ in $C$, then $\displaystyle \limsup_{n \uparrow \infty} f(\mathbf{x}_n) \leq f(\mathbf{x}^*)$. 
\end{defn}

\begin{cor}
    If $f \colon C \subset \mathbb{R}^d \to \mathbb{R}$ is lower semicontinuous, $C$ is closed and bounded, then $f$ attains its minimum.
\end{cor}
\begin{proof}
    Let $\left\{ \mathbf{x}_n \right\} \in C$ be such that $f(\mathbf{x}_n) \downarrow \beta \vcentcolon= \inf_{\mathbf{x} \in C} f(\mathbf{x})$. By Bolzano-Weierstrass, $\exists \left\{ \mathbf{x}_{n_k} \right\}$ such that $\mathbf{x}_{n_k} \to \mathbf{x}^*$. Since $C$ is closed, $\mathbf{x}^* \in C$. Then, 
    \[
        \beta = \lim_{n \uparrow \infty} f(\mathbf{x}_{n_k}) = \liminf_{n \uparrow \infty} f(\mathbf{x}_{n_k}) \geq f(\mathbf{x}^*) \geq \beta \implies f(\mathbf{x}^*) = \beta.
    \]
    Similarly, an upper semicontinuous function attains its maximum on a closed and bounded domain.
\end{proof}

\begin{prop}
    Let $g \colon C \times D \to \mathbb{R}$ where $C \subset \mathbb{R}^n$, $D \subset \mathbb{R}^m$. Define $f \colon C \to \mathbb{R}$ as
    \[
        f(\mathbf{x}) \vcentcolon= \sup_{\mathbf{y} \in D} g(\mathbf{x}, \mathbf{y}) \quad (\text{resp. } \inf_{\mathbf{y} \in D} g(\mathbf{x}, \mathbf{y}))
    \]
    Suppose $f(\mathbf{x}) < \infty$ (resp. $f(\mathbf{x}) > -\infty$) for all $\mathbf{x} \in C$. If $g(\mathbf{x},\mathbf{y})$ is continuous in $\mathbf{x}$ for all $\mathbf{y} \in D$, then $f$ is lower semicontinuous (resp. upper semicontinuous).
\end{prop}
\begin{proof}
    Let $\mathbf{x}_n \to \mathbf{x}^*$ in $C$. Then, 
    \begin{align*}
        \liminf_{n \uparrow \infty} f(\mathbf{x}_n) &\geq \liminf_{n \uparrow \infty} g(\mathbf{x}_n, \mathbf{y}) \, \forall \mathbf{y} \in D \\
        &= \lim_{n \uparrow \infty} g(\mathbf{x}_n, \mathbf{y}) \\
        &= g(\mathbf{x}^*, \mathbf{y}).
    \end{align*}
    Thus, 
    \begin{align*}
        \liminf_{n \uparrow \infty} f(\mathbf{x}_n) &\geq g(\mathbf{x}^*, \mathbf{y}) \quad \forall \mathbf{y} \in D \\
        \implies \liminf_{n \uparrow \infty} f(\mathbf{x}_n) &\geq \sup_{\mathbf{y} \in D} g(\mathbf{x}^*, \mathbf{y}) = f(\mathbf{x}^*). \qedhere
    \end{align*}
\end{proof}
\end{document}

