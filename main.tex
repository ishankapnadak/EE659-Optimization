\documentclass[12pt]{article}
\usepackage{amsmath, amssymb, amsfonts, amsthm, mathtools,mathrsfs}
\usepackage{bm}
\usepackage{thmtools}
\usepackage[utf8]{inputenc}
\usepackage[inline]{enumitem}
\usepackage[colorlinks=true]{hyperref}
\usepackage{multicol}
\usepackage{witharrows}
\usepackage{tikz}
\usetikzlibrary{automata,positioning}
\usetikzlibrary{decorations.markings}
\usepackage{verbatim}
\usetikzlibrary{arrows.meta} 
\usepackage{witharrows}
\usepackage[useregional, showdow]{datetime2}
\usepackage{physics}
\DTMlangsetup[en-GB]{abbr}
\usepackage{xcolor}
\usepackage[normalem]{ulem}
\usetikzlibrary{chains,shapes.multipart}
\usepackage{algorithm}
\usepackage{algpseudocode}

\usepackage{bbm}
\usepackage[page,toc,titletoc,title]{appendix}
\usepackage{tocloft}
\setlength\parindent{0pt}
\usepackage{parskip}

\def\D{\mathrm{d}}
\def\I{\mathbb{I}}
\def\P{\mathbb{P}}
\def\E{\mathbb{E}}
\def\Var{\text{Var}}
\def\F{\mathcal{F}} 
\def\ccdf{\overline{F}} 
\newcommand*{\thead}[1]{\multicolumn{1}{c}{\bfseries #1}}
\renewcommand{\arraystretch}{1}
\newcommand{\indep}{\perp \!\!\! \perp}

\usepackage[framemethod=tikz]{mdframed}
\mdfdefinestyle{theoremstyle}{%
	% linecolor=gray,linewidth=1pt,%
	% frametitlerule=true,%
	frametitlebackgroundcolor=white,
	% backgroundcolor=  gray!20,	
	bottomline=false, topline=false, rightline=false, leftline=true,
	innerlinewidth=0.7pt, outerlinewidth=0.2pt, middlelinewidth=2pt, middlelinecolor=white, %
	innerleftmargin=6pt,
	% innertopmargin=-1pt,
	skipabove=10pt,
	% fontcolor=blue,
	% innerbottommargin=-0.5pt,
}
\mdtheorem[style=theoremstyle]{defn}[thm]{Definition}[section]
\mdtheorem[style=theoremstyle]{lem}[thm]{Lemma}
\mdtheorem[style=theoremstyle]{prop}[thm]{Proposition}
\mdtheorem[style=theoremstyle]{thm}{Theorem}[section]
\mdtheorem[style=theoremstyle]{cor}{Corollary}[section]


\newcommand*{\doublerule}{\hrule width \hsize height 1pt \kern 0.5mm \hrule width \hsize height 2pt}
\newcommand{\doublerulefill}{\leavevmode\leaders\vbox{\hrule width .1pt\kern1pt\hrule}\hfill\kern0pt}
\def\ddfrac#1#2{\displaystyle\frac{\displaystyle #1}{\displaystyle #2}}
\newcommand{\interior}[1]{%
  {\kern0pt#1}^{\mathrm{o}}%
}

%\newcommand{\Res}{\operatorname{Res}}

\theoremstyle{definition}
% \numberwithin{thm}{section}
% \newtheorem{lem}[thm]{Lemma}
% \newtheorem{defn}[thm]{Definition}
% \newtheorem{prop}[thm]{Proposition}
% \newtheorem{cor}[thm]{Corollary}
% \newtheorem{ex}{Example}


\let\emptyset\varnothing

\usepackage{titlesec}
\titleformat{\section}[block]{\Large\filcenter\bfseries}{\S\thesection.}{0.25cm}{\Large}
\titleformat{\subsection}[block]{\large\bfseries\sffamily}{\S\S\thesubsection.}{0.2cm}{\large}

\usepackage[a4paper]{geometry}
\usepackage{lipsum}
\usepackage{xcolor,cancel}

\usepackage{cleveref}
\crefname{thm}{Theorem}{Theorems}
\crefname{lem}{Lemma}{Lemmas}
\crefname{defn}{Definition}{Definitions}
\crefname{prop}{Proposition}{Propositions}
\crefname{cor}{Corollary}{Corollaries}
\crefname{equation}{}{}
\DeclareMathOperator*{\argmin}{arg\,min}
\DeclareMathOperator*{\Vol}{Vol}
\DeclareMathOperator*{\co}{co}
\DeclareMathOperator*{\cco}{\overline{co}}
\DeclareMathOperator*{\diag}{diag}
\DeclareMathOperator*{\Span}{span}
\DeclareMathOperator*{\epi}{epi}

\usepackage{mdframed}
\newenvironment{blockquote}
{\begin{mdframed}[skipabove=0pt, skipbelow=0pt, innertopmargin=4pt, innerbottommargin=4pt, bottomline=false,topline=false,rightline=false, linewidth=2pt]}
{\end{mdframed}}
\newenvironment{soln}{\begin{proof}[Solution]}{\end{proof}}

\title{A First Course in Optimization\\}
\author{Ishan Kapnadak}
\date{Autumn Semester 2022-23\\~\\Updated on: \textcolor{blue}{\DTMToday}}

\begin{document}
\tikzset{lab dis/.store in=\LabDis,
  lab dis=-0.4,
  ->-/.style args={at #1 with label #2}{decoration={
    markings,
    mark=at position #1 with {\arrow{>}; \node at (0,\LabDis) {#2};}},postaction={decorate}},
  -<-/.style args={at #1 with label #2}{decoration={
    markings,
    mark=at position #1 with {\arrow{<}; \node at (0,\LabDis)
    {#2};}},postaction={decorate}},
  -*-/.style args={at #1 with label #2}{decoration={
    markings,
    mark=at position #1 with {{\fill (0,0) circle (1.5pt);} \node at (0,\LabDis)
    {#2};}},postaction={decorate}},
  }
\maketitle

\begin{abstract}
    \begin{center}
        Lecture Notes for the course EE 659: A First Course in Optimization taught in Spring 2022 by Prof. Vivek Borkar. Additional references include \textit{A first course in optimization} by Rangarajan K. Sundaram, \textit{Optimization by vector space methods} by David Luenberger, and \textit{Nonlinear programming} by Dimitri P. Bertsekas
    \end{center}
\end{abstract}

\tableofcontents
\newpage

\section{Lecture 1}
\medskip

\begin{defn}[Open Ball]
    The \emph{open ball} of radius $\epsilon$ centered around $\mathbf{x}_0 \in \mathbb{R}^d$ is defined as
    \[
        B_{\epsilon}(\mathbf{x}_0) \vcentcolon= \left\{ \mathbf{x} \in \mathbb{R}^d \colon \norm{\mathbf{x} - \mathbf{x}_0} < \epsilon \right\}.
    \]
\end{defn}
\begin{defn}[Closed Ball]
    The \emph{closed ball} of radius $\epsilon$ centered around $\mathbf{x}_0 \in \mathbb{R}^d$ is defined as
    \[
        \overline{B}_{\epsilon}(\mathbf{x}_0) \vcentcolon= \left\{ \mathbf{x} \in \mathbb{R}^d \colon \norm{\mathbf{x} - \mathbf{x}_0} \leq \epsilon \right\}.
    \]
\end{defn}
\begin{defn}[Open and Closed Sets]
    A set $A \subset \mathbb{R}^d$ is said to be \emph{open} if for all $\mathbf{x} 
    \in A$, there exists an $\epsilon > 0$ such that $B_{\epsilon}(\mathbf{x}) \subset A$. A set $A$ is said to be \emph{closed} if $A^{\mathsf{c}}$ is open.
\end{defn}

We have the following properties for open and closed sets.

\begin{enumerate}
    \item Let $\mathcal{I}$ be an arbitrary index set. If $A_{\alpha}$ is open for each $\alpha \in \mathcal{I}$, then 
    \[
        \bigcup_{\alpha \in \mathcal{I}} \, A_{\alpha}
    \]
    is open. In other words, open sets are closed under arbitrary unions.

    \item Let $\mathcal{I}$ be a finite index set. If $A_{\alpha}$ is open for each $\alpha \in \mathcal{I}$, then 
    \[
        \bigcap_{\alpha \in \mathcal{I}} \, A_{\alpha}
    \]
    is open. In other words, open sets are closed under finte intersections.

    \item Let $\mathcal{I}$ be an arbitrary index set. If $A_{\alpha}$ is closed for each $\alpha \in \mathcal{I}$, then 
    \[
        \bigcap_{\alpha \in \mathcal{I}} \, A_{\alpha}
    \]
    is closed. In other words, closed sets are closed under arbitrary intersections.

    \item Let $\mathcal{I}$ be a finite index set. If $A_{\alpha}$ is closed for each $\alpha \in \mathcal{I}$, then 
    \[
        \bigcup_{\alpha \in \mathcal{I}} \, A_{\alpha}
    \]
    is closed. In other words, closed sets are closed under finite unions.
\end{enumerate}

\begin{defn}[Convergence of a sequence]
    Let $\langle \mathbf{x}_n \rangle$ be a sequence in $\mathbb{R}^d$. Then, $\langle\mathbf{x}_n\rangle$ converges to $\mathbf{x}^*$ (written $\mathbf{x}_n \to \mathbf{x}^*$) if for all $\epsilon > 0$, there exists an $n_0 \in \mathbb{N}$ such that
    \[
        \mathbf{x}_n \in B_{\epsilon}(\mathbf{x}^*) \quad \forall n > n_0.
    \]
    Equivalently, $\norm{\mathbf{x}_n - \mathbf{x}^*} \to 0$.
\end{defn}

\begin{defn}[Closure]
    Let $A \subset \mathbb{R}^d$. The \emph{closure} of $A$ (denoted $\overline{A}$) is the smallest closed set containing $A$. Equivalently, $\overline{A}$ is the intersection of all closed sets containing $A$.
\end{defn}

\begin{defn}[Interior]
    Let $A \subset \mathbb{R}^d$. The \emph{interior} of $A$ (denoted $\interior{A}$) is the largest open set contained in $A$. Equivalently, $\interior{A}$ is the union of all open sets contained in $A$.
\end{defn}
Note that by definition, we have $\interior{A} \subset A \subset \overline{A}$.

\begin{defn}[Boundary]
    Let $A \subset \mathbb{R}^d$. The \emph{boundary} of $A$ is defined as
    \[
        \partial A \vcentcolon= \overline{A} \setminus \interior{A}.
    \]
\end{defn}
Note that for a closed set, $\overline{A} = A$, and for an open set, $\interior{A} = A$. 

\begin{prop}
    A set $A \subset \mathbb{R}^d$ is closed if and only if
    \[
        \mathbf{x}_n  \to \mathbf{x}^*, \mathbf{x}_n \in A \forall n \implies \mathbf{x}^* \in A.
    \]
\end{prop}
\begin{proof}
    Let $A \subset \mathbb{R}^d$ be closed, $\mathbf{x}_n \in A$ for all $n$, and $\mathbf{x}_n \to \mathbf{x}^*$. Assume to the contrary that $\mathbf{x}^* \notin A$. Then, $\mathbf{x}^* \in A^{\mathsf{c}}$, which is open by assumption. Thus, $\exists \epsilon > 0$ such that $B_{\epsilon}(\mathbf{x}^*) \subset A^{\mathsf{c}}$. This implies that $\mathbf{x}_n \notin B_{\epsilon}(\mathbf{x}^*)$ for all $n$, and thus $\mathbf{x}_n \not\to \mathbf{x}^*$, a contradiction. To prove the converse, assume that $A$ is not closed. Thus, there exists a $\Tilde{\mathbf{x}} \in \partial A$ such that $\Tilde{\mathbf{x}} \notin A$. Then, for all $\epsilon > 0$, $B_{\epsilon}(\Tilde{\mathbf{x}}) \cap A \neq \emptyset$. Let $\epsilon_n \downarrow 0$ and let $\mathbf{x}_n \in B_{\epsilon_n}(\Tilde{\mathbf{x}}) \cap A$. Then, $\mathbf{x}_n \to \Tilde{\mathbf{x}} \notin A$, a contradiction.
\end{proof}

\begin{defn}[Limit Point]
    Let $\langle \mathbf{x}_n \rangle$ be a sequence in $\mathbb{R}^d$. $\Tilde{\mathbf{x}}$ is a \emph{limit point} of $\langle \mathbf{x}_n \rangle$ if there exists a subsequence $\langle \mathbf{x}_{n_k} \rangle$ such that $\mathbf{x}_{n_k} \to \Tilde{\mathbf{x}}$.
\end{defn}

\begin{prop}
    $\langle \mathbf{x}_n \rangle$ converges if and only if $\langle \mathbf{x}_n \rangle$ has a unique limit point.    
\end{prop}

\begin{defn}[Supremum and Infimum]
    Let $A \subset \mathbb{R}$ be bounded. Then, 
    \begin{align*}
        \sup A &\vcentcolon= \text{ smallest } x \in \mathbb{R} \cup \{+\infty\} \text{ such that } y \in A \implies y \leq x, \\
        \inf A &\vcentcolon= \text{ largest } x \in \mathbb{R} \cup \{-\infty\} \text{ such that } y \in A \implies y \geq x.
    \end{align*}
\end{defn}

\begin{defn}[Cauchy Sequence]
    A sequence $\langle \mathbf{x}_n \rangle$ is said to be \emph{Cauchy} if $\displaystyle\lim_{m,n \uparrow \infty} \norm{\mathbf{x}_m - \mathbf{x}_n} = 0$.
\end{defn}

\begin{prop}
    Cauchy sequences are bounded.
\end{prop}
\begin{proof}
    Let $\langle \mathbf{x}_n \rangle$ be a Cauchy sequene and let $\epsilon > 0$. Pick $N$ large enough such that 
    \[
        n,m > N \implies \norm{\mathbf{x}_m - \mathbf{x}_n} < \epsilon.
    \]
    We then have
    \begin{align*}
        &\mathbf{x}_n \in B_{\epsilon}(\mathbf{x}_m) \quad \forall n > N \\
        &\implies \langle \mathbf{x}_n \colon n > N\rangle \text{ is bounded} \\
        &\implies \langle \mathbf{x}_n \rangle \text{ is bounded.} \qedhere
    \end{align*}
\end{proof} 
\begin{prop}
    Cauchy sequences have at most one limit point.
\end{prop}
\begin{proof}
    Suppose $\langle \mathbf{x}_n \rangle$ is a Cauchy sequence having two limit points, $\Tilde{\mathbf{x}}$ and $\overline{\mathbf{x}}$. Then, there exist subsequences $\mathbf{x}_{\Tilde{n}_k} \to \Tilde{\mathbf{x}}$ and $\mathbf{x}_{\overline{n}_l} \to \overline{\mathbf{x}}$. We then have
    \[
        \lim_{\Tilde{n}_k, \overline{n}_l \uparrow \infty} \norm{\mathbf{x}_{\Tilde{n}_k} - \mathbf{x}_{\overline{n}_l}} = 0 \implies \Tilde{\mathbf{x}} = \overline{\mathbf{x}}. \qedhere
    \]
\end{proof}
\begin{defn}[Complete Space]
    A metric space is \emph{complete} if every Cauchy sequence converges.
\end{defn}

\begin{thm}[Bolzano-Weierstrass Theorem]
    Every bounded sequence has a convergent subsequence.
\end{thm}
\newpage
\section{Lecture 2}

\begin{thm}[Bolzano-Weierstrass Theorem]
    Every bounded sequence ]oin $\mathbb{R}^d$ has a convergent subsequence.
\end{thm}
\begin{proof}
    Suppose $d = 1$. Since $\left\{ x_n \right\}$ is bounded, we have $x_n \in [a,b]$ for all $n$ where $a,b \in \mathbb{R}$ and $a < b$. The idea is to keep halving the interval and pick a half interval containing infinitely many points. For example, consider the two half intervals $[a, \frac{a+b}{2}]$ and $[\frac{a+b}{2}, b]$. Since $\left\{ x_n \right\}$ has infinitely many points, at least one of these two half intervals has infinitely many points. Call this half interval $[a_1, b_1]$ and repeat this argument again for $[a_1, b_1]$. This gives us a sequence $\left\{ (a_n, b_n) \right\}$ satisfying 
    \begin{align*}
        a_0 \leq a_1 \leq a_2 \leq \cdots b &\implies a_n \to a^* \\
        b_0 \geq b_1 \geq b_2 \geq \cdots a &\implies b_n \to b^* \\
    \end{align*}
    where we define $a_0 \vcentcolon= a$ and $b_0 \vcentcolon= b$. Moreover, we have
    \[
        \abs{b_n - a_n} = \frac{b-a}{2^n} \to 0 \implies a^* = b^*.
    \]
    Since there are infinitely many $x_n$'s in $[a_k, b_k]$ for any $k$, pick $\Tilde{x}_k \in [a_k,b_k] \cap \{x_n\}$ such that $\Tilde{x}_k \neq \Tilde{x}_j$ for $j < k$. Thus, $\Tilde{x}_k \to a^* = b^*$. This can be generalised to $d > 1$ via induction and we leave this as an exercise to the reader.
\end{proof}

Note that the above argument does not generalise to infinite dimensions. For example, consider the complete orthonormal space
\[
    \mathcal{L}_2[0,T] \vcentcolon= \left\{ f \colon [0,T] \to \mathbb{R} \colon \int_0^T f^2(t) \, \D t < \infty \right\}
\]
with inner product
\[
    \langle f,g \rangle \vcentcolon= \int_0^T f(t) g(t) \, \D t.
\]
Consider an orthonormal basis $\{e_n\}$ satisfying 
\[
    \langle e_n, e_m \rangle = 
    \begin{cases}
        1 & \text{ if } n = m, \\
        0 & \text{ if } n \neq m.
    \end{cases}
\]
Note that $\norm{e_n - e_m} = \sqrt{2}$ whenever $n \neq m$ and thus $\{e_n\}$ has no convergent subsequence. 

\begin{prop}
    Let $f \colon C \subset \mathbb{R}^d \to \mathbb{R}$ be bounded from below. Let $\beta = \inf_{\mathbf{x} \in C} f(\mathbf{x})$. Then, $\exists \left\{ \mathbf{x}_n \right\} \in C$ such that $f(\mathbf{x}_n) \downarrow \beta$.  
\end{prop}

\begin{thm}[Weierstrass Theorem]
    Let $C \subset \mathbb{R}^d$ be closed and bounded, and let $f \colon C \to \mathbb{R}$ be continuous. Then, $f$ attains its minimum and maximum. 
\end{thm}
\begin{proof}
    Let $\left\{ \mathbf{x}_n \right\rangle \in C$ be such that $f(\mathbf{x}_n) \downarrow \beta \vcentcolon= \inf_{\mathbf{x} \in C} f(\mathbf{x})$. By Bolzano-Weierstrass, $\exists \left\{ \mathbf{x}_{n_k} \right\}$ such that $\mathbf{x}_{n_k} \to \mathbf{x}^*$. Since $C$ is closed, $\mathbf{x}^* \in C$. Since $f$ is continuous, $f(\mathbf{x}_{n_k}) \to f(\mathbf{x}^*) \implies f(\mathbf{x}^*) = \beta$. A similar argument holds for maximum.
\end{proof}

\begin{cor}
    Let $C \subset \mathbb{R}^d$ be closed and let $f \colon C \to \mathbb{R}$ satisfy 
    \[
        \lim_{\norm{\mathbf{x}} \uparrow \infty} f(\mathbf{x}) = \infty.
    \]
    Then, $f$ attains its minimum on $C$. 
\end{cor}
\begin{proof}
    Let $\left\{ \mathbf{x}_n \right\}$ be such that $f(\mathbf{x}_n) \downarrow \beta \vcentcolon= \inf_{\mathbf{x} \in C} f(\mathbf{x})$. Then, $\left\{ \mathbf{x}_n \right\}$ is bounded, since otherwise $\exists \left\{ \mathbf{x}_{n_k} \right\}$ such that $\norm{\mathbf{x}_{n_k}} \uparrow \infty \implies f(\mathbf{x}_{n_k}) \to \infty \neq \beta$. The previous argument now follows through.
\end{proof}

\begin{defn}[Limit Supremum and Limit Infimum]
    Let $\left\{ x_n \right\} \in \mathbb{R}$. We define
    \begin{align*}
        \limsup_{n \uparrow \infty} x_n &\vcentcolon= \lim_{n \uparrow \infty} \, \sup_{m \geq n} x_m = \inf_{n \geq 1} \, \sup_{m \geq n} x_m \\
        \liminf_{n \uparrow \infty} x_n &\vcentcolon= \lim_{n \uparrow \infty} \, \inf_{m \geq n} x_m = \sup_{n \geq 1} \, \inf_{m \geq n} x_m
    \end{align*}
    We sometimes also denote the limit supremum as $\overline{\lim} x_n$ and the limit infimum as $\underline{\lim} x_n$.
\end{defn}

Note that $\limsup$ and $\liminf$ are always well-defined if we allow $\{\pm\infty\}$ as possibilities. This is because $\sup_{m \geq n} x_m$ is a non-increasing sequence and thus must converge (possibly to $-\infty$). Similarly, $\inf_{m \geq n} x_m$ is a non-decreasing sequence and thus must converge (possibly to $+\infty$). We also note that

\begin{enumerate}
    \item $\displaystyle \limsup_{n \uparrow \infty} x_n \geq \liminf_{n \uparrow \infty} x_n$.

    \item If $\displaystyle \limsup_{n \uparrow \infty} x_n = \liminf_{n \uparrow \infty} x_n = x^*$, then $x_n \to x^*$. 
\end{enumerate}

\begin{defn}[Lower and Upper Semicontinuous]
    $f \colon C \subset \mathbb{R}^d \to \mathbb{R}$ is said to be lower semicontinuous (l.s.c) if whenever $\mathbf{x}_n \to \mathbf{x}^*$ in $C$, then $\displaystyle \liminf_{n \uparrow \infty} f(\mathbf{x}_n) \geq f(\mathbf{x}^*)$. 

    $f \colon C \subset \mathbb{R}^d \to \mathbb{R}$ is said to be upper semicontinuous (u.s.c) if whenever $\mathbf{x}_n \to \mathbf{x}^*$ in $C$, then $\displaystyle \limsup_{n \uparrow \infty} f(\mathbf{x}_n) \leq f(\mathbf{x}^*)$. 
\end{defn}

\begin{cor}
    If $f \colon C \subset \mathbb{R}^d \to \mathbb{R}$ is lower semicontinuous, $C$ is closed and bounded, then $f$ attains its minimum.
\end{cor}
\begin{proof}
    Let $\left\{ \mathbf{x}_n \right\} \in C$ be such that $f(\mathbf{x}_n) \downarrow \beta \vcentcolon= \inf_{\mathbf{x} \in C} f(\mathbf{x})$. By Bolzano-Weierstrass, $\exists \left\{ \mathbf{x}_{n_k} \right\}$ such that $\mathbf{x}_{n_k} \to \mathbf{x}^*$. Since $C$ is closed, $\mathbf{x}^* \in C$. Then, 
    \[
        \beta = \lim_{n \uparrow \infty} f(\mathbf{x}_{n_k}) = \liminf_{n \uparrow \infty} f(\mathbf{x}_{n_k}) \geq f(\mathbf{x}^*) \geq \beta \implies f(\mathbf{x}^*) = \beta.
    \]
    Similarly, an upper semicontinuous function attains its maximum on a closed and bounded domain.
\end{proof}

\begin{prop}
    Let $g \colon C \times D \to \mathbb{R}$ where $C \subset \mathbb{R}^n$, $D \subset \mathbb{R}^m$. Define $f \colon C \to \mathbb{R}$ as
    \[
        f(\mathbf{x}) \vcentcolon= \sup_{\mathbf{y} \in D} g(\mathbf{x}, \mathbf{y}) \quad (\text{resp. } \inf_{\mathbf{y} \in D} g(\mathbf{x}, \mathbf{y}))
    \]
    Suppose $f(\mathbf{x}) < \infty$ (resp. $f(\mathbf{x}) > -\infty$) for all $\mathbf{x} \in C$. If $g(\mathbf{x},\mathbf{y})$ is continuous in $\mathbf{x}$ for all $\mathbf{y} \in D$, then $f$ is lower semicontinuous (resp. upper semicontinuous).
\end{prop}
\begin{proof}
    Let $\mathbf{x}_n \to \mathbf{x}^*$ in $C$. Then, 
    \begin{align*}
        \liminf_{n \uparrow \infty} f(\mathbf{x}_n) &\geq \liminf_{n \uparrow \infty} g(\mathbf{x}_n, \mathbf{y}) \, \forall \mathbf{y} \in D \\
        &= \lim_{n \uparrow \infty} g(\mathbf{x}_n, \mathbf{y}) \\
        &= g(\mathbf{x}^*, \mathbf{y}).
    \end{align*}
    Thus, 
    \begin{align*}
        \liminf_{n \uparrow \infty} f(\mathbf{x}_n) &\geq g(\mathbf{x}^*, \mathbf{y}) \quad \forall \mathbf{y} \in D \\
        \implies \liminf_{n \uparrow \infty} f(\mathbf{x}_n) &\geq \sup_{\mathbf{y} \in D} g(\mathbf{x}^*, \mathbf{y}) = f(\mathbf{x}^*). \qedhere
    \end{align*}
\end{proof}
\newpage
\section{Lecture 3}

\begin{defn}[Directional Derivative]
    Let $f \colon \mathbb{R}^d \to \mathbb{R}$ and $\mathbf{h} \in \mathbb{R}^d$ be a unit vector. We say that $f$ is differentiable at $\mathbf{x}$ in direction $\mathbf{h}$ if
    \[
        \lim_{\epsilon \downarrow 0} \frac{f(\mathbf{x}+\epsilon\mathbf{h}) - f(\mathbf{x})}{\epsilon}
    \]
    exists. In this case, the limit is called the \emph{directional derivative} of $f$ at $\mathbf{x}$ in direction $\mathbf{h}$, and is denoted $f^{\prime}(\mathbf{x},\mathbf{h})$.
\end{defn}

\begin{defn}[Gateaux Derivative]
    Let $f \colon \mathbb{R}^d \to \mathbb{R}$ and $\mathbf{h} \in \mathbb{R}^d$ be a unit vector. If the limit
    \[
        \lim_{\epsilon \to 0} \frac{f(\mathbf{x}+\epsilon\mathbf{h}) - f(\mathbf{x})}{\epsilon}
    \]
    exists then it is called the \emph{Gateaux derivative} of $f$ at $\mathbf{x}$ along the line $\{\epsilon \mathbf{h} \colon \epsilon \in \mathbb{R}\}$, and is denoted $f^{\prime}(\mathbf{x},\mathbf{h})$. Alternatively, we may write 
    \[
    f(\mathbf{x} + \epsilon\mathbf{h}) = f(\mathbf{x}) + \epsilon f^{\prime}(\mathbf{x},\mathbf{h}) + o_{\mathbf{h}}(\epsilon)
    \]
    where $ \frac{o_{\mathbf{h}(\epsilon)}}{\epsilon} \to 0$ as $\epsilon \to 0$.
\end{defn}

\begin{defn}[Fréchet Derivative]
    Let $f \colon \mathbb{R}^d \to \mathbb{R}$. If there exists a linear map $D_{\mathbf{x}}f \colon \mathbb{R}^d \to \mathbb{R}$ such that
    \[
        \sup_{\norm{\mathbf{h}} = 1} \, \norm{\frac{f(\mathbf{x}+\epsilon\mathbf{h}) - f(\mathbf{x})}{\epsilon} - D_{\mathbf{x}}(\mathbf{h})} \to 0
    \]
    as $\epsilon \to 0$, then $f$ is said to be \emph{Fréchet differentiable} and $D_{\mathbf{x}}f$ is called its Fréchet derivative at $\mathbf{x}$.
\end{defn}

If $f \colon \mathbb{R}^d \to \mathbb{R}$, then $D_{\mathbf{x}}f \in \mathbb{R}^d$ and is called the \emph{gradient}, denoted as $\nabla f(\mathbf{x})$. 

If $f \colon \mathbb{R}^d \to \mathbb{R}^d$, then $D_{\mathbf{x}}f \in \mathbb{R}^{d \times d}$ and is called the \emph{Jacobian matrix}, denoted as
\[
    D_{\mathbf{x}}f = \left[\left[ \frac{\partial f_i}{\partial x_j}\right]\right]_{1 \leq i,j \leq d}.
\]
Now, $\nabla f(\cdot) \colon \mathbb{R}^d \to \mathbb{R}^d$. Its derivative is the Jacobian matrix of $\nabla f$ and is called the \emph{Hessian} of $f$, denoted as
\[
    \nabla^2 f(\mathbf{x}) = \left[\left[ \frac{\partial^2 f}{\partial x_i \partial x_j}\right]\right]_{1 \leq i,j \leq d}.
\]

Now onwards, unless otherwise mentioned, when we say that a function is differentiable, we mean that it is Fréchet differentiable.

\begin{prop}
    If $f$ is differentiable at $\mathbf{x}_0$, then 
    \[
        f(\mathbf{x}) = f(\mathbf{x}_0) + \left\langle \nabla f(\mathbf{x}_0, \mathbf{x} - \mathbf{x}_0 \right\rangle + o(\norm{\mathbf{x} - \mathbf{x}_0}).
    \]
    If $f$ is twice differentiable at $\mathbf{x}_0$, then 
    \[
        f(\mathbf{x}) = f(\mathbf{x}_0) + \left\langle \nabla f(\mathbf{x}_0, \mathbf{x} - \mathbf{x}_0 \right\rangle + \frac{1}{2} (\mathbf{x} - \mathbf{x}_0)^{top} \nabla^2 f(\mathbf{x}_0) (\mathbf{x} - \mathbf{x}_0) + o(\norm{\mathbf{x} - \mathbf{x}_0}^2).
    \]
\end{prop}

\begin{thm}
    If $f$ is differentiable and has a local minimum at $\mathbf{x}_0$, then $\nabla f(\mathbf{x}_0) = \mathbf{0}$.
\end{thm}

\begin{thm}
    If $\mathbf{x}_0$ is a strict local minimum (i.e. there exists an open neighbourhood $O$ of $\mathbf{x}_0$ such that $f(\mathbf{y}) > f(\mathbf{x}_0)$ for all $\mathbf{y} \in O \setminus \{\mathbf{x}_0\}$) and $f$ is differentiable, then $\nabla^2f(\mathbf{x}_0)$ is positive semidefinite. Conversely, if $\nabla f(\mathbf{x}_0) = \mathbf{0}$ and $\nabla^2f(\mathbf{x}_0)$ is positive definite, then $\mathbf{x}_0$ is a local minimum.
\end{thm}

We now move to the setting of a generalized constrained optimization problem. Suppose $C \subseteq \mathbb{R}^d$ is open and $f \colon C \to \mathbb{R}$ is continuously differentiable. Suppose that $g_1, \ldots, g_k, h_1, \ldots, h_s \colon \mathbb{R}^d \to \mathbb{R}$ are all continuously differentiable ($k,s \geq 1$). We consider the constrained optimization problem
\[
    \min_{\mathbf{x} \in C} f(\mathbf{x})
\]
subject to
\begin{align*}
    g_i(\mathbf{x}) &= 0 \text{ for all } i \in \{1,\ldots,k\}, \quad \text{(equality constraints)} \\
    h_i(\mathbf{x}) &\leq 0 \text{ for all } i \in \{1,\ldots,s\}. \quad \text{(inequality constraints)}
\end{align*}

\begin{thm}
    Suppose $\mathbf{x}_0 \in C$ satisfies the constraints and $f(\mathbf{x}_0) \leq f(\mathbf{x})$ for all $\mathbf{x} \in C$. Then, there exist $\lambda_0, \lambda_1, \ldots, \lambda_k, \mu_1, \ldots, \mu_s$ such that
    \[
        \lambda_0 \frac{\partial f}{\partial x_j}(\mathbf{x}_0) + \sum_{i=1}^k \lambda_i \frac{\partial g_i}{\partial x_j}(\mathbf{x}_0) + \sum_{i=1}^s \mu_i \frac{\partial h_i}{\partial x_j}(\mathbf{x}_0) = 0 \quad \forall j. 
    \]
    Furthermore, 
    \begin{enumerate}
        \item $\lambda_0 \geq 0, \mu_r \geq 0$ for all $r$,
        \item $h_r(\mathbf{x}_0) < 0 \implies \mu_r = 0$ (complementary slackness), and
        \item if $\nabla g_i(\mathbf{x}_0)$ ($1 \leq i \leq k$) and those $\nabla h_r(\mathbf{x}_0)$ for which $h_r(\mathbf{x}_0)$ are linearly independent, then $\lambda_0 = 1$ without loss of generality.
    \end{enumerate}
\end{thm}

\begin{proof}
    Without loss of generality, let $\mathbf{x}_0 = \mathbf{0}$ and $f(\mathbf{x}_0) = 0$. Assume that $h_i(\mathbf{x}_0) = 0$ for $1 \leq i \leq l$, and $h_i(\mathbf{x}_0) < 0$ for $l < i \leq s$. Pick $\epsilon^*$ such that $\overline{B}_{\epsilon^*}(\mathbf{0}) \subseteq C$ and $h_i(\mathbf{x}) < 0$ for all $i > l$ and $\mathbf{x} \in \overline{B}_{\epsilon^*}(\mathbf{0})$.

    \begin{lem}
        For all $\epsilon \in (0,\epsilon^*)$, there exists $N_{\epsilon} \geq 1$ such that
        \[
            f(\mathbf{x}) + \norm{\mathbf{x}}^2 + N_{\epsilon} \left( \sum_{i=1}^k g_i(\mathbf{x})^2 + \sum_{j=1}^l h_j^+(\mathbf{x})^2 \right) > 0 \quad \forall \mathbf{x} \in \partial B_{\epsilon}(\mathbf{0}),
        \]
        where $h_j^+(\mathbf{x}) \vcentcolon= \max(0, h_j(\mathbf{x}))$.
    \end{lem}  

    \begin{proof}
        Assume to the contrary. Then, $\exists N_m \uparrow \infty$ and $\mathbf{x}_m \in \partial B_{\epsilon}(\mathbf{x})$, $m \geq 1$ such that
        \[
            f(\mathbf{x}_m) + \norm{\mathbf{x}_m}^2 \leq -N_m \left( \sum_{i=1}^k g_i(\mathbf{x}_m)^2 + \sum_{j=1}^l h_j^+(\mathbf{x}_m)^2 \right) \quad \forall m.
        \]
        By Bolzano-Weierstrass, $\mathbf{x}_m \to \mathbf{x}^*$ along a subsequence denoted by $\{\mathbf{x}_m\}$ again, so that $f(\mathbf{x}_m) \to f(\mathbf{x}^*)$ by continuity. Dividing both sides by $-N_m$ and letting $m \uparrow \infty$, we get
        \[
            \sum_{i=1}^k g_i(\mathbf{x}^*)^2 + \sum_{j=1}^l h_j^+(\mathbf{x}^*)^2 \leq 0.
        \]
        Thus, $g_i(\mathbf{x}^*) = 0$ for all $1 \leq i \leq k$, and $h_j(\mathbf{x}^*) \leq 0$ for all $1 \leq j \leq l$. $\mathbf{x}^*$ satisfies the constraints and thus $f(\mathbf{x}^*) \geq f(\mathbf{x}_0)$. Note that $f(\mathbf{x}_m) \leq -\epsilon^2 \implies f(\mathbf{x}^*) \leq 0-\epsilon^2$. But, $f(\mathbf{x}_0) = 0$, a contradiction.
    \end{proof}

    Now, we define
    \[
        F(\mathbf{x}) \vcentcolon= f(\mathbf{x}) + \norm{x}^2 + N_{\epsilon} \left( \sum_{i=1}^k g_i(\mathbf{x})^2 + \sum_{j=1}^l h_j^+(\mathbf{x})^2 \right).
    \]
    Let $\hat{\mathbf{x}}$ be a minimizer of $F(\cdot)$ on $\overline{B}_{\epsilon}(\mathbf{0})$. Then, $F(\hat{\mathbf{x}}) \leq F(\mathbf{0}) = 0$. Thus, $\hat{\mathbf{x}} \notin \partial B_{\epsilon}(\mathbf{0})$ since we showed that $F$ is positive on $\partial B_{\epsilon}(\mathbf{0})$. Thus, $\nabla F(\hat{\mathbf{x}}) = \mathbf{0}$. Evaluating the derivative, we have
    \[
        \frac{\partial f}{\partial x_j}(\hat{\mathbf{x}}) + 2\hat{x}_j + 2N_{\epsilon} \sum_{i=1}^k g_i(\hat{\mathbf{x}}) \frac{\partial g_i}{\partial x_j}(\hat{\mathbf{x}}) + 2N_{\epsilon} \sum_{i=1}^l h_i(\hat{\mathbf{x}}) \frac{\partial h_i}{\partial x_j}(\hat{\mathbf{x}}) = 0 \quad \forall j.
    \]

    Next, we put $\epsilon^* = \epsilon^m \downarrow 0$ and rewrite as $N^m, \hat{\mathbf{x}}^m$, etc. The above equation can then be rewritten as
    \[
        \lambda_0^m \frac{\partial f}{\partial x_j}(\hat{\mathbf{x}}^m) + \frac{2\hat{\mathbf{x}}^m}{\norm{\mathbf{z}^m}} + \sum_{i=1}^k \lambda_i^m \frac{\partial g_i}{\partial x_j}(\hat{\mathbf{x}}^m) + \sum_{i=1}^l \mu_i^m \frac{\partial h_i}{\partial x_j}(\hat{\mathbf{x}}^m) = 0 \quad \forall j,
    \]
    where
    \[
        \mathbf{z}^m = \begin{bmatrix}
            1 & 2N^m g_1(\hat{\mathbf{x}}^m) & \cdots & 2N^m g_k(\hat{\mathbf{x}}^m) & 2N^m h_1^+(\hat{\mathbf{x}}^m) & \cdots & 2N^m h_l^+(\hat{\mathbf{x}}^m) & 0 & \cdots & 0
        \end{bmatrix} \in \mathbb{R}^{1+k+s}
    \]
    and we divide throughout by $\norm{\mathbf{z}^m}$. Further, we let
    \[
        \mathbf{U}^m \vcentcolon= \begin{bmatrix}
            \lambda_0^m & \lambda_1^m & \cdots & \lambda_k^m & \mu_1^m & \cdots \mu_l^m & 0 & \cdots & 0
        \end{bmatrix}
    \]
    with $\norm{\mathbf{U}^m} = 1$. Thus, $\mathbf{U}^m$ is the unit vector in the direction of $\mathbf{z}^m$. By the Bolzano-Weierstrass Theorem,
    \[
        \mathbf{U}^m \to \begin{bmatrix}
            \lambda_0 & \lambda_1 & \cdots & \lambda_k & \mu_1 & \cdots \mu_l & 0 & \cdots & 0
        \end{bmatrix}
    \]
    along a subsequence, and $\hat{\mathbf{x}}^m \to \mathbf{x}_0$. Thus, we get
    \[
        \lambda_0 \frac{\partial f}{\partial x_j}(\mathbf{x}_0) + \sum_{i=1}^k \lambda_i \frac{\partial g_i}{\partial x_j}(\mathbf{x}_0) + \sum_{i=1}^l \mu_i^m \frac{\partial h_i}{\partial x_j}(\mathbf{x}_0) = 0 \quad \forall j.
    \]
\end{proof}

With $\lambda_0 \geq 0$, the above theorem is called the Fritz-John condition, whereas with $\lambda_0 = 1$ the above theorem is known as the famous Karush-Kuhn-Tucker condition. 
\newpage
\section{Lecture 4}

\begin{thm}[Envelope Theorem / Danskin's Theorem]
    Suppose $C \subseteq \mathbb{R}^d$ is open and $D \subseteq \mathbb{R}^m$ is closed and bounded. Suppose that $f \colon C \times D \to \mathbb{R}$ is continuous and its partial gradient with respect to $\mathbf{x} \in C$, denoted as
    \[
        \nabla^{\mathbf{x}} f(\mathbf{x}, \mathbf{y}) \vcentcolon= \begin{bmatrix}
            \frac{\partial f}{\partial x_1}(\mathbf{x}) & \cdots & \frac{\partial f}{\partial x_d}(\mathbf{x})
        \end{bmatrix} \in \mathbb{R}^d
    \]
    is continuous. Let $g(\mathbf{x}) \vcentcolon= \max_{\mathbf{y} \in D} \, f(\mathbf{x}, \mathbf{y})$, where the maximum is attained on a non-empty closed and bounded set $M(\mathbf{x}) \subseteq D$. Then, $g \colon \mathbb{R}^d \to \mathbb{R}$ has a directional derivative in every direction, given by
    \[
        g^{\prime}(\mathbf{x} ; \mathbf{n}) = \max_{\mathbf{y} \in M(\mathbf{x})} \, \left\langle \nabla^{\mathbf{x}} f(\mathbf{x}, \mathbf{y}), \mathbf{n} \right\rangle
    \]
    for every unit vector $\mathbf{n} \in \mathbb{R}^d$. 
\end{thm}

\begin{defn}[Convex Set]
    A set $C \subseteq \mathbb{R}^d$ is said to be \emph{convex} if for all $\mathbf{x}, \mathbf{y} \in C$, we have
    \[
        \alpha\mathbf{x} + (1-\alpha)\mathbf{y} \in C \quad \forall \alpha \in [0,1].
    \]
\end{defn}

\begin{prop}
    A set $C \subseteq \mathbb{R}^d$ is convex iff $\forall \mathbf{x}_1, \ldots \mathbf{x}_n \in C$, $n > 0$ and $\alpha_i > 0$ with $\sum_i \alpha_i = 1$, we have
    \[
        \sum_{i=1}^n \alpha_i \mathbf{x}_i \in C.
    \]
\end{prop}
\begin{proof}
    $(\impliedby)$ is clear by taking $n = 2$. For $(\implies)$, we already know that the result holds for $n = 2$, by definition. Suppose the statement holds for some $n \geq 2$. Then, 
    \[
        \sum_{i=1}^{n+1} \alpha_i \mathbf{x}_i = \alpha_1 \mathbf{x}_1 + (1-\alpha_1) \cdot \underbrace{\sum_{i=2}^{n+1} \frac{\alpha_i}{1-\alpha_1} \mathbf{x}_i}_{\in C} \in C,
    \]
    where the latter point is in $C$ by the induction hypothesis. The result then follows from induction.
\end{proof}
Note that if $C$ is closed, it suffices to check that $\mathbf{x}, \mathbf{y} \in C \implies \frac{\mathbf{x} + \mathbf{y}}{2} \in C$. 

We list out some properties of convex sets below. 

\begin{enumerate}
    \item Convex sets are connected.
    \item Intersection of an arbitrary family of convex sets is convex.
    \item Union of two convex sets need not be convex.
    \item Interior and closure of convex sets are convex.
    \item Image of a convex set under an affine map is convex.
\end{enumerate}

\begin{defn}[Convex Hull]
    Let $A \subseteq \mathbb{R}^d$. The \emph{convex hull} of $A$, denoted $\co(A)$, is the smallest convex set containing $A$, or equivalently, the intersection of all convex sets containing $A$, or equivalently, the set of convex combinations of all points in $A$. 
\end{defn}

\begin{defn}[Closed Convex Hull]
    Let $A \subseteq \mathbb{R}^d$. The \emph{closed convex hull} of $A$, denoted $\cco(A)$, is the smallest closed convex set containing $A$, or equivalently, the intersection of all closed convex sets containing $A$.
\end{defn}

\begin{thm}
    Let $C \subseteq \mathbb{R}^d$ be closed and convex and let $\mathbf{x} \notin C$. Then, there exists a unique $\mathbf{x}^* \in C$ such that
    \[
        \norm{\mathbf{x} - \mathbf{x}^*} = \min_{\mathbf{y} \in C} \, \norm{\mathbf{x} - \mathbf{y}}.
    \]
\end{thm}

\begin{proof}
    Note that $\mathbf{y} \mapsto \norm{\mathbf{x} - \mathbf{y}}$ is a continuous map. By triangle inequality, $\norm{\mathbf{x} - \mathbf{y}} \geq \norm{\mathbf{y}} - \norm{\mathbf{x}}$, and thus
    \[
        \lim_{\norm{\mathbf{y}} \uparrow \infty} \norm{\mathbf{x} - \mathbf{y}} = \infty.
    \]
    The existence of a minimizer $\mathbf{x}^*$ now follows from Corollary 2.1 of the Weierstrass Theorem. Suppose $\hat{\mathbf{x}} \neq \mathbf{x}^*$ is another minimizer. Then, the triangle formed by $(\mathbf{x}, \mathbf{x}^*, \hat{\mathbf{x}})$ is an isosceles triangle with line segment $(\mathbf{x}^*, \hat{\mathbf{x}})$ as its base. Moreover, this triangle lies completely in $C$ by convexity. By elementary geometry, it is easy to see that the midpoint $\frac{\mathbf{x}^* + \hat{\mathbf{x}}}{2}$ is at a strictly smaller distance from $\mathbf{x}$ than $\mathbf{x}^*, \hat{\mathbf{x}}$, a contradiction. Thus, $\mathbf{x}^*$ is the unique minimizer of $\norm{\mathbf{y} - \mathbf{x}}$ over $C$. 
\end{proof}

The above $\mathbf{x}^*$ is called the \emph{projection} of $\mathbf{x}$ onto $C$. This theorem immediately adapts to a more general setting, as follows.

\begin{thm}
    Let $C,D \subseteq \mathbb{R}^d$ be disjoint closed convex sets with $C$ bounded. Then, there exist $\mathbf{x}^* \in C$, $\mathbf{y}^* \in D$ such that
    \[
        0 < \norm{\mathbf{x}^* - \mathbf{y}^*} = \min_{\mathbf{x} \in C , \mathbf{y} \in D} \, \norm{\mathbf{x} - \mathbf{y}}.
    \]
\end{thm}
\begin{proof}
    Consider the map $(\mathbf{x}, \mathbf{y}) \in C \times D \mapsto \norm{\mathbf{x} - \mathbf{y}}$, which is clearly continuous. If $D$ is also bounded, the claim is immediate from Weierstrass Theorem (Theorem 2.3). If not, we have
    \[
        \norm{(\mathbf{x}, \mathbf{y})} \uparrow \infty \iff \norm{\mathbf{y}} \uparrow \infty,
    \]
    since $C$ is bounded. Thus, $\norm{(\mathbf{x}, \mathbf{y})} \uparrow \infty \implies \norm{\mathbf{x} - \mathbf{y}} \uparrow \infty$. The existence of a minimizing pair $(\mathbf{x}^*, \mathbf{y}^*)$ again follows from Corollary 2.1 of the Weierstrass Theorem. Moreover, since $C,D$ are disjoint, $\norm{\mathbf{x}^* - \mathbf{y}^*} > 0$.
\end{proof}
Note that no uniqueness can be claimed. (Hint: imagine two disjoint rectangles in $\mathbb{R}^2$ with sides parallel to each other). Next, we characterize the projection of $\mathbf{x}$ onto $C$.

\begin{thm}
    Let $C \subseteq \mathbb{R}^d$ be closed convex and let $\mathbf{x} \notin C$. Then,
    \[
        \mathbf{x}^* = \argmin_{\mathbf{y} \in C} \norm{\mathbf{x} - \mathbf{y}}
    \]
    if and only if
    \[
        \left\langle \mathbf{y} - \mathbf{x}^*, \mathbf{x} - \mathbf{x}^* \right\rangle \leq 0.
    \]
\end{thm}
\begin{proof}
    The proof is left as an exercise. 
\end{proof}
\newpage
\section{Lecture 5}

\begin{defn}
    A \emph{hyperplane} with normal vector $\mathbf{n}$ and passing through point $\mathbf{x}_0 \in \mathbb{R}^d$ is defined as the set
    \[
        \mathcal{H} \vcentcolon= \left\{ \mathbf{x} \in \mathbb{R}^d \colon \left\langle \mathbf{x} - \mathbf{x}_0, \mathbf{n} \right\rangle = 0 \right\}.
    \]
    $\mathcal{H}$ also defines two \emph{closed half spaces}
    \begin{align*}
        L_{\mathcal{H}} &\vcentcolon= \left\{ \mathbf{x} \in \mathbb{R}^d \colon \left\langle \mathbf{x} - \mathbf{x}_0, \mathbf{n} \right\rangle \leq 0 \right\} \\
        U_{\mathcal{H}} &\vcentcolon= \left\{ \mathbf{x} \in \mathbb{R}^d \colon \left\langle \mathbf{x} - \mathbf{x}_0, \mathbf{n} \right\rangle \geq 0 \right\}
    \end{align*}
    that intersect in $\mathcal{H}$.
\end{defn}

\begin{defn}
    Let $A,B \subseteq \mathbb{R}^d$ and let $\mathcal{H}$ be a hyperplane in $\mathbb{R}^d$. We say that $\mathcal{H}$ \emph{separates} $A,B$ if $A \subseteq L_{\mathcal{H}}$ and $B \subseteq U_{\mathcal{H}}$ or vice-versa. 
\end{defn}

\begin{thm}
    \begin{enumerate}
        \item If $C \subseteq \mathbb{R}^d$ is closed convex and $\mathbf{x} \notin C$, then there exists a hyperplane separating the two. 
        \item If $C,D \subseteq \mathbb{R}^d$ are disjoint closed convex sets, then there exists a hyperplane separating the two. 
    \end{enumerate}
\end{thm}
\begin{proof}
    For the first part, it suffices to take $\mathbf{n} \vcentcolon= (\mathbf{x} - \mathbf{x}^*)/\norm{\mathbf{x} - \mathbf{x}^*}$ and $\mathbf{x}_0 = \mathbf{x}^*$. We leave the proof for this part as an exercise. (Hint: Use Theorem 4.7).
\end{proof}

\begin{defn}[Extreme Point]
    Let $C \subseteq \mathbb{R}^d$ be convex. A point $\mathbf{x} \in C$ is said to be an \emph{extreme point} of $C$ if it cannot be expressed as a strict convex combination of two distinct points in $C$. That is,
    \[
        \mathbf{x} = \alpha \mathbf{y} + (1-\alpha) \mathbf{z}, \, \mathbf{y}, \mathbf{z} \in C, \, \alpha \in (0,1) \implies \mathbf{x} = \mathbf{y} = \mathbf{z}.
    \]
\end{defn}
Clearly, an extreme point $\mathbf{x} \in \partial C$ because if not, there is an open ball centered at $\mathbf{x}$ completely contained in $C$. 

\begin{thm}
    A closed bounded convex set $C \subseteq \mathbb{R}^d$ has an extreme point. 
\end{thm}
\begin{proof}
    Let $\mathbf{l}_1, \ldots, \mathbf{l}_d$ be linearly independent vectors in $\mathbb{R}^d$. Let $C_0 \vcentcolon= C$ and for $1 \leq i < d$, recursively define
    \[
        C_{i+1} \vcentcolon= \left\{ \mathbf{x} \in C_i \colon \left\langle \mathbf{l}_i, \mathbf{x} \right\rangle = \min_{\mathbf{y} \in C_i} \left\langle \mathbf{l}_i, \mathbf{y} \right\rangle \right\}.
    \]
    Then, $C_{i+1} \subseteq C_i$ for all $i$, and $C_i$'s are closed bounded and convex. If $\mathbf{x}, \mathbf{y} \in C_d$, then $\langle \mathbf{l}_i, \mathbf{x}\rangle = \langle \mathbf{l}_i, \mathbf{y} \rangle$ for all $i$, implying that $\mathbf{x} = \mathbf{y}$ by our choice of $\mathbf{l}_i$'s. Thus, $C_d = \{ \mathbf{x}^* \}$ for some $\mathbf{x}^* \in C$. We claim that $\mathbf{x}^*$ is an extreme point of $C$. If not, we may choose $\mathbf{y} \neq \mathbf{z}$, both in $C$, such that $\mathbf{x}^* = \frac{\mathbf{y} + \mathbf{z}}{2}$. Then,
    \[
        \left\langle \mathbf{l}_1, \mathbf{x} \right\rangle + \frac{1}{2}\left\langle \mathbf{l}_1, \mathbf{y} \right\rangle + \frac{1}{2}\left\langle \mathbf{l}_1, \mathbf{z} \right\rangle
    \]
    implying that $\left\langle \mathbf{l}_1, \mathbf{x} \right\rangle = \left\langle \mathbf{l}_1, \mathbf{y} \right\rangle = \left\langle \mathbf{l}_1, \mathbf{z} \right\rangle$, and thus $\mathbf{y}, \mathbf{z} \in C_1$. Repeating this argument, we conclude that $\mathbf{y}, \mathbf{z} \in C_i$ for all $i$, and in particular $\mathbf{y}, \mathbf{z} = C_d = \{ \mathbf{x}^* \}$. The claim thus follows. 
\end{proof}

\begin{thm}
    A closed and convex set has an extreme point if and only if it contains no lines. 
\end{thm}
\begin{proof}
    Let $C \subseteq \mathbb{R}^d$ be closed and convex. Suppose $C$ contains a line $\left\{ \overline{\mathbf{x}} + t\mathbf{h} \colon t \in \mathbb{R} \right\}$ passing through $\overline{\mathbf{x}} \in \mathbb{R}^d$ in the direction $\mathbf{h} \in \mathbb{R}^d$. Then, $\mathbf{x} + t\mathbf{h} \in C$ for any $t \in \mathbb{R}, \mathbf{x} \in C$. Indeed, 
    \[
        \mathbf{x} + t\mathbf{h} = \lim_{\epsilon \to 0} \left[ (1-\epsilon) \mathbf{x} + \epsilon \left( \overline{\mathbf{x}} + \frac{t}{\epsilon} \mathbf{h} \right) \right] \in C
    \]
    since $C$ is closed and convex. Thus, no point of $C$ can be an extreme point. 

    Conversely, assume that $C$ has no lines. We use induction to prove that $C$ has an extreme point. If $C$ is a closed, convex subset of $\mathbb{R}$ having no lines, then $C$ is a closed and bounded interval and thus has an extreme point. Now, assume the statement holds for dimensions strictly less than $d$, and consider a closed convex set $C \subseteq \mathbb{R}^d$. Since $C$ has no lines, $C$ has a boundary point, say $\overline{\mathbf{x}}$. Let $\mathcal{H}$ be the supporting hyperplane of $C$ at $\overline{\mathbf{x}}$. Now, $\mathcal{H} \cap C$ lies in $(d-1)$-dimensional space. Since it contains no lines, it has an extreme point by the induction hypothesis. It is easy to see that this extreme point is also an extreme point of $C$. (In fact, we shall prove this in a later lecture). 
\end{proof}

\begin{thm}[Krein-Milman Theorem in Finite Dimensions]
    A closed bounded convex set is the closed convex hull of its extreme points.
\end{thm}
\begin{proof}
    Let $C \subseteq \mathbb{R}^d$ be closed, convex, and bounded, and let $\mathcal{E}(C)$ denote the set of its extreme points. It is clear that $\cco(\mathcal{E}(C)) \subseteq C$. Suppose $\cco(\mathcal{E}(C)) \neq C$, then $\exists \mathbf{x}^* \in \cco(\mathcal{E}(C)) \setminus C$. We can construct a support hyperplane $\mathcal{H} = \left\{ \mathbf{x} \colon \left\langle \mathbf{x} - \mathbf{x}_0, \mathbf{n} \right\rangle = 0 \right\}$ of $\cco(\mathcal{E}(C))$ such that $\mathcal{E}(C) \subseteq L_{\mathcal{H}}$ and $\mathbf{x}^* \in \interior{U_{\mathcal{H}}}$. Then, $\langle \mathbf{x}^* - \mathbf{x}_0, \mathbf{n} \rangle > 0$. As in the proof of Theorem 5.5, take $\mathbf{l}_1 \vcentcolon= -\mathbf{n}$ to obtain $\hat{\mathbf{x}} \in C$ with $\langle \hat{\mathbf{x}^*} - \mathbf{x}_0, \mathbf{n} \rangle > 0$, a contradiction. The claim follows. 
\end{proof}
\newpage
\section{Lecture 6}

\begin{lem}
    Let $C \subseteq \mathbb{R}^d$ be closed, bounded, and convex, and let $\mathcal{H}$ be a support hyperplane of $C$ such that $C \subseteq L_{\mathcal{H}}$. Then, $C_1 \vcentcolon= C \cap \mathcal{H}$ is closed bounded convex, and $\mathcal{E}(C_1) \subseteq \mathcal{E}(C)$. 
\end{lem}
\begin{proof}
    The first claim follows directly from the fact that $C$ is closed bounded convex and $\mathcal{H}$ is closed convex. Now, if $\mathcal{E}(C_1) \not\subseteq \mathcal{E}(C)$, then $\exists \mathbf{x} \in \mathcal{E}(C_1) \setminus \mathcal{E}(C)$. Then, $\mathbf{x} = \alpha\mathbf{y} + (1-\alpha)\mathbf{z}$ for some $\alpha \in (0,1)$ and $\mathbf{y}, \mathbf{z} \in \partial C$ with $\mathbf{y} \neq \mathbf{z}$. Clearly, at least one of $\mathbf{y}, \mathbf{z}$ (say $\mathbf{y}$) is not in $C_1$. Then, it is in $\partial C \setminus \mathcal{H} \subseteq \interior{L_{\mathcal{H}}}$. Then, $\mathbf{z} \in \interior{U_{\mathcal{H}}} \cap \partial C$ which is empty. Hence, $\mathbf{y}, \mathbf{z} \in \mathcal{H} \subseteq C_1$, but then $\mathbf{x} \notin \mathcal{E}(C_1)$, a contradiction.
\end{proof}

\begin{thm}[Carathéodory's Theorem]
    Let $C \subseteq \mathbb{R}^d$ be closed bounded convex. Then, any $\mathbf{x} \in C$ can be written as a convex combination of at most $(d+1)$ extreme points of $C$. 
\end{thm}
\begin{proof}
    We prove this by induction on $d$. For $d = 1$, $C$ is a closed bounded interval, say $[a,b]$ which has extreme points $a$ and $b$. For any $x \in [a,b]$, we may write
    \[
        x = \frac{b-x}{b-a} \cdot a + \frac{x-a}{b-a} \cdot b
    \]
    which is a convex combination of exactly $2$ extreme points. Suppose the claim holds for some $d \geq 1$. Let $C \subseteq \mathbb{R}^{d+1}$ be closed bounded and convex, with non-empty interior. If $\mathbf{x} \in \mathcal{E}(C)$, the $\mathbf{x}$ can clearly be written as a trivial convex combination of $1$ extreme point. Else, take $\mathbf{e}_1 \in \mathcal{E}(C)$ and extend the line segment from $\mathbf{e}_1$ through $\mathbf{x}$ till the point $\mathbf{b}_1 \vcentcolon= \mathbf{e}_1 + a(\mathbf{x} - \mathbf{e}_1)$ where $a \geq 1$ is the maximum number for which $\mathbf{b}_1$ thus defined lies in $C$. Then, $\mathbf{b}_1 \in \partial C$ (it is possible that $\mathbf{x} \in \partial C$ in which case $a = 1$ and $\mathbf{b}_1 = \mathbf{x}$). Let $\mathcal{H}_1$ be a support hyperplane at $\mathbf{b}_1$ such that $C \subseteq L_{\mathcal{H}_1}$. Let $C_1 \vcentcolon= C \cap \mathcal{H}_1$ which is closed bounded and convex as well. By induction hypothesis, $\mathbf{b}_1$ is a convex combination of at most $(d+1)$ extreme points in $\mathcal{E}(C_1) \subseteq \mathcal{E}(C)$. That is, we may write
    \[
        \mathbf{b}_1 = \sum_{i=2}^{d+2} \alpha_i \mathbf{e}_i
    \]
    for $\alpha_i \in [0,1]$ with $\mathbf{e}_2, \ldots, \mathbf{e}_{d+2} \in \mathcal{E}(C)$ and $\sum_i \alpha_i = 1$. Note that since we need not require all $d+1$ extreme points, some $\alpha_i$'s may be zero. Now, we have
    \begin{align*}
        \mathbf{b}_1 = \mathbf{e}_1 + a(\mathbf{x} - \mathbf{e}_1) &\implies \mathbf{x} = \mathbf{e}_1 + \frac{1}{a} \mathbf{b_1} - \frac{1}{a} \mathbf{e}_1 \\
        &\implies \mathbf{x} = \frac{a-1}{a} \cdot \mathbf{e}_1 + \sum_{i=2}^{d+2} \frac{\alpha_i}{a} \cdot \mathbf{e}_i
    \end{align*}
    which is a convex combination of at most $(d+2)$ points in $\mathcal{E}(C)$. 
\end{proof}

\begin{thm}[Dubin's Theorem]
    Let $C \subseteq \mathbb{R}^d$ be closed bounded convex and let $H_1, \ldots, H_m$ be closed half-spaces in $\mathbb{R}^d$ with $m \leq d$, defined as
    \[
        H_i \vcentcolon= \left\{ \mathbf{x} \in \mathbb{R}^d \colon \left\langle \mathbf{r}_i, \mathbf{x} \right\rangle \leq c_i \right\}, \quad 1 \leq i \leq m.
    \]
    Let 
    \[
        C^* \vcentcolon= C \cap \left( \bigcap_{i=1}^m H_i \right).
    \]
    Then, any extreme points of $C^*$ can be written as a convex combination of at most $(m+1)$ extreme points of $C$. 
\end{thm}
\begin{proof}
    Suppose not, then there is some $\mathbf{x} \in \mathcal{E}(C^*)$ that can be written as a strict convex combination of $k$ elements of $\mathcal{E}(C)$ (with $m+1 < k \leq d+1$), say $\mathbf{x}_1, \ldots, \mathbf{x}_k$ and no less. These $\mathbf{x}_i$'s must form a $(k-1)$-simplex $\Delta \subseteq C$, with $\mathbf{x} \in \interior{\Delta}$. Clearly, $\mathbf{x} \notin \mathcal{E}(C)$. Consider a closed ball $\overline{B} \subseteq \Delta$ centered at $\mathbf{x}$. The intersection of $\overline{B}$ with $0 \leq l \leq m$ hyperplanes passing through $\mathbf{x}$ will be a disc $B^{\prime}$ centered at $\mathbf{x}$ such that the dimension of $\Span\{ \mathbf{y} - \mathbf{x} \colon \mathbf{y} \in B^{\prime} \}$ is at least $l$. Thus, $\mathbf{x}$ cannot be in $\mathcal{E}(C^*)$, a contradiction. 
\end{proof}

In several applications, we encounter a set of the form
\[
    \mathcal{K} \vcentcolon= \left\{ \mathbf{x} \in \mathbb{R}^d \colon \mathbf{Ax} \leq \mathbf{b} \right\}
\]
where $\mathbf{A} \in \mathbb{R}^{m \times d}$ and $\mathbf{b} \in \mathbb{R}^m$. We discuss the structure of extreme points of $\mathcal{K}$. 

\newpage

\begin{thm}
    Let $\mathbf{x} \in \mathcal{K}$. Then, $\mathbf{x}$ is an extreme point of $\mathcal{K}$ if and only if some $d$ inequalities corresponding to $d$ linearly independent rows of the system $\mathbf{Ax} \leq \mathbf{b}$ are equalities. That is, $\langle \mathbf{a}_i, \mathbf{x} \rangle = b_i$ for $i$ corresponding to those $d$ linearly independent rows of $\mathbf{A}$.
\end{thm}
\begin{proof}
    Let $\mathbf{x} \in \mathcal{K}$ be an extreme point. Let
    \[
        \mathcal{I} = \left\{ i \in \{1, \ldots, m \} \colon \langle \mathbf{a}_i, \mathbf{x} \rangle = b_i \right\}.
    \]
    Let $\mathcal{F} = \{ \mathbf{a}_i \colon i \in \mathcal{I} \}$. We need to show that $\mathcal{F}$ contains $d$ linearly independent vectors, or equivalently, that $\Span (\mathcal{F}) \supseteq \mathbb{R}^d$. Suppose $\Span(\mathcal{F}) \subsetneq \mathbb{R}^d$. Thus, $\mathcal{F}^{\perp} \neq \{\mathbf{0}\}$. Choose $\mathbf{z} \neq \mathbf{0} \in \mathcal{F}^{\perp}$. For $i \neq \mathcal{I}$, 
    \[
        \left\langle \mathbf{a}_i, \mathbf{x} \pm \epsilon \mathbf{z} \right\rangle = \left\langle \mathbf{a}_i, \mathbf{x} \right\rangle \pm \epsilon \left\langle \mathbf{a}_i, \mathbf{z} \right\rangle \leq b_i
    \]
    for $\epsilon$ sufficiently small, since $\langle \mathbf{a}_i, \mathbf{x} \rangle < b_i$. Thus, $\mathbf{x} \pm \epsilon \mathbf{z} \in \mathcal{K}$ for such $\epsilon$. But then,
    \[
        \mathbf{x} = \frac{1}{2} (\mathbf{x} + \epsilon\mathbf{z}) + \frac{1}{2} (\mathbf{x} - \epsilon\mathbf{z})
    \]
    contradicting the fact that $\mathbf{x}$ is an extreme point. Thus, $\Span(\mathcal{F}) \supseteq \mathbb{R}^d$. 

    Conversely, let $\mathbf{x} \in \mathcal{K}$ be such that $d$ of the inequalities of the system $\mathbf{Ax} \leq \mathbf{b}$ are equalities. If $\mathbf{x}$ is not an extreme point, then there exist $\mathbf{z} \in \mathcal{K}$ such that $\mathbf{x} \pm \mathbf{z} \in \mathcal{K}$. Then,
    \begin{align*}
        \left\langle \mathbf{a}_i, \mathbf{x} + \mathbf{z} \right\rangle \leq b_i &\implies b_i + \left\langle \mathbf{a}_i, \mathbf{z} \right\rangle \leq b_i \implies \left\langle \mathbf{a}_i, \mathbf{z} \right\rangle \leq 0 \\
        \left\langle \mathbf{a}_i, \mathbf{x} - \mathbf{z} \right\rangle \leq b_i &\implies b_i - \left\langle \mathbf{a}_i, \mathbf{z} \right\rangle \leq b_i \implies \left\langle \mathbf{a}_i, \mathbf{z} \right\rangle \geq 0
    \end{align*}
    Thus, $\langle \mathbf{a}_i, \mathbf{z} \rangle = 0$ for each $i$ for which $\langle \mathbf{a}_i, \mathbf{x} \rangle$. Note that there are at least $d$ such linearly vectors $\mathbf{a}_i$. Since $\mathbf{z} \in \mathbb{R}^d$, it follows that $\mathbf{z} = \mathbf{0}$, and hence, $\mathbf{x}$ is an extreme point of $\mathcal{K}$.
\end{proof}

\begin{cor}
    $\mathcal{K}$ has finitely many extreme points.
\end{cor}

Finally, we state some facts about extreme points without proof. 

\begin{enumerate}
    \item A boundary point may not be an extreme point. (For example, the points on the edge of a rectangle except the corners are not extreme points.)

    \item A convex set may not have an extreme point if it is not closed. (For example, the open ball in $\mathbb{R}^d$ has no extreme points.)

    \item If $\mathbf{x} \in \partial C$ is not an extreme point, and $\mathbf{y}, \mathbf{z} \in C$ are such that $\mathbf{x} = \alpha \mathbf{y} + (1-\alpha)\mathbf{z}$ for some $\alpha \in (0,1)$, then $\mathbf{y}, \mathbf{z} \in \partial C$. 

    \item An extreme point may have a unique supporting hyperplane that contains other non-extreme points. 

    \item The set of extreme points of a closed bounded convex set can be written as a countable intersection of open sets. 
\end{enumerate}

\begin{defn}[Convex Function]
    Let $C \subseteq \mathbb{R}^d$ be convex. A function $f \colon C \to \mathbb{R}$ is said to be \emph{convex} if for all $\mathbf{x}, \mathbf{y} \in C$, $\alpha \in [0,1]$, we have
    \[
        f\left( \alpha \mathbf{x} + (1-\alpha)\mathbf{y} \right) \leq \alpha f(\mathbf{x}) + (1-\alpha) f(\mathbf{y}).
    \]
\end{defn}
\begin{prop}
    Let $C \subseteq \mathbb{R}^d$ be convex. A function $f \colon C \to \mathbb{R}$ is convex iff $\forall\mathbf{x}_1, \ldots, \mathbf{x}_n \in C$, $a_i \geq 0$ with $\sum_i a_i =1$, we have
    \[
        f\left( \sum_{i=1}^n \alpha_i \mathbf{x}_i \right) \leq \sum_{i=1}^n \alpha_i f(\mathbf{x}_i).
    \]
\end{prop}
\begin{proof}
    Hint: Follow the proof for Proposition 4.3.
\end{proof}

The following properties follow directly from the definition and thus, we state them without proof. 

\begin{enumerate}
    \item Positive linear combinations of convex functions are convex. That is, if $\left\{ f_i \colon 1 \leq i \leq n \right\}$ are convex, and $a_1, \ldots, a_n \geq 0$, then $\sum_{i=1}^n a_i f_i$ is convex.

    \item Pointwise limits of convex functions are convex. That is, if $f_n$ is convex for each $n \geq 1$, and $f(\mathbf{x}) \vcentcolon= \lim_{n \uparrow \infty} f_n(\mathbf{x})$ exists for each $\mathbf{x} \in C$, then $f$ is convex. 

    \item Pointwise suprema or maxima of convex functions are convex whenever well-defined (that is, whenever pointwise finite).

    \item Pointwise infima or minima of convex functions need not be convex.

    \item If $f_1, \ldots, f_m$ are convex, and $g \colon \mathbb{R}^m \to \mathbb{R}$ is convex and componentwise increasing, then $h(\cdot) \vcentcolon= g\left( f_1(\cdot), \ldots, f_m(\cdot) \right) \colon C \to \mathbb{R}$ is convex. 

    \item Let $\{ f_\alpha \})$ be a family of convex functions indexed by a parameter $\alpha \in \mathbb{R}^m$ and let $\varphi \colon \mathbb{R}^m \to [0,\infty)$. Suppose
    \[
        f^*(\mathbf{x}) \vcentcolon= \int f_{\alpha}(\mathbf{x}) \varphi(\alpha) \, \mathrm{d}\alpha
    \]
    is well-defined as a Riemann integral for all $\mathbf{x} \in C$. Then, $f^*$ is convex.

    \item Let $f \colon \mathbb{R}^d \to \mathbb{R}$ be convex, $\mathbf{b} \in \mathbb{R}^d$, $\mathbf{A} \in \mathbb{R}^{d \times m}$. Then, $g \colon \mathbb{R}^m \to \mathbb{R}$ defined as $g(\mathbf{x}) \vcentcolon= f(\mathbf{Ax} + \mathbf{b})$ is convex. 
\end{enumerate}
\newpage
\section{Lecture 7}

\begin{defn}[Epigraph]
    Let $C \subseteq \mathbb{R}^d$ be convex and let $f \colon \mathbb{R}$ be a function. The \emph{epigraph} of $f$ is defined as the set of points on or above the graph of $f$. That is, 
    \[
        \epi(f) \vcentcolon= \left\{ (\mathbf{x}, y) \in \mathbb{R}^d \times \mathbb{R} \colon y \geq f(\mathbf{x}) \right\}.
    \]
\end{defn}

\begin{lem}
    $f$ is a convex function if and only if $\epi(f)$ is a convex set.
\end{lem}
\begin{thm}
    Let $f \colon C \to \mathbb{R}$ be convex for a convex $C \subseteq \mathbb{R}^d$ with a non-empty interior. Then, $f$ is continuous at any $\mathbf{x}_0 \in \interior{C}$. 
\end{thm}
\begin{thm}
    Let $f \colon \mathbb{R}^d \to \mathbb{R}$ be convex. Then, $f$ is differentiable almost everywhere
\end{thm}

\begin{defn}[Lipschitz and Locally Lipschitz]
    A function $f \colon \mathbb{R}^d \to \mathbb{R}$ is said to be \emph{locally Lipschitz} if for any bounded open set $B \subseteq \mathbb{R}^d$, there exists a constant $K_B > 0$ (possibly depending on $B$) such that for all $\mathbf{x}, \mathbf{y} \in B$, we have
    \[
        \abs{f(\mathbf{x}) - f(\mathbf{y})} \leq K_B \cdot \norm{\mathbf{x} - \mathbf{y}}.
    \]
    If this constant $K_B$ does not depend on $B$, then $f$ is said to be \emph{Lipschitz}.
\end{defn}

\begin{thm}[Rademacher's Theorem]
    A Lipschitz function is differentiable almost everywhere.
\end{thm}

\begin{thm}[Alexandrov's Theorem]
    A convex function $f \colon \mathbb{R}^d \to \mathbb{R}$ is locally Lipschitz and twice differentiable almost everywhere.
\end{thm}

\begin{thm}
    \begin{enumerate}
        \item Let $f \colon C \to \mathbb{R}$ be convex, where $C \subseteq \mathbb{R}^d$ is convex and open. Then, for any $\mathbf{x} \in C$ where $f$ is differentiable, we have
        \[
            f(\mathbf{y}) \geq f(\mathbf{x}) +  \left\langle \nabla f(\mathbf{x}), \mathbf{y} - \mathbf{x} \right\rangle \quad \forall \mathbf{y} \in C.
        \]
        Conversely, if $f \colon C \to \mathbb{R}$ is continuously differentiable and the above holds for all $\mathbf{x}, \mathbf{y} \in C$, then $f$ is convex. 

        \item If $f$ as defined above is twice continuously differentiable at $\mathbf{x} \in C$, then $\nabla^2f(\mathbf{x})$ is positive semi-definite. Conversely, if $\nabla^2f(\mathbf{x})$ is positive semi-definite in $C$, then $f$ must be convex. 
    \end{enumerate}
\end{thm}

\begin{thm}
    Let $f \colon \mathbb{R}^d \to \mathbb{R}$ be convex. Then, 
    \[
        f(\mathbf{x}) = \sup \left\{ g(\mathbf{x}) \colon g \text{ is affine and } g \leq f \right\}.
    \]
\end{thm}

\begin{thm}
    Let $C \subseteq \mathbb{R}^d$ be closed convex and bounded and $f \colon C \to \mathbb{R}$ be convex continuous. Then, $\epi(f)$ is a closed set. 
\end{thm}
\newpage
\section{Lecture 8}

The next result we prove is the Lagrange multiplier rule, the crown jewel of convex optimization. We look at minimizing a continuous convex function $f \colon C \to \mathbb{R}$ defined on a closed convex set $C \subseteq \mathbb{R}^d$, subject to constraints
\[
    g_i(\mathbf{x}) \leq 0 , \quad 1 \leq i \leq m,
\]
where $g_i \colon C \to \mathbb{R}$ is continuous for each $i$. In particular, the set of `feasible' points
\[
    \Tilde{C} \vcentcolon= C \cap \left( \bigcap_{i=1}^m \left\{ \mathbf{x} \in C \colon g_i(\mathbf{x}) \leq 0 \right\} \right)
\]
is a closed, convex set. We may compactly write the constraints as $g(\mathbf{x}) \leq \mathbf{0}$ where
\[
    g(\mathbf{x}) \vcentcolon= \begin{bmatrix}
        g_1(\mathbf{x}) \\ \vdots \\ g_m(\mathbf{x})
    \end{bmatrix} \in \mathbb{R}^m,
\]
and the inequalities are component-wise. Consider the set
\[
    D \vcentcolon= \left\{ (\mathbf{y}, z) \in \mathbb{R}^m \times \mathbb{R} \colon \exists \, \mathbf{x} \in \mathbb{R}^d \text{ such that } y \geq g(\mathbf{x}) \text{ and } z \geq f(\mathbf{x}) \right\}.
\]
It is easy to see that $D$ is convex. (Why?) 

\begin{lem}
    Define $w \colon \mathbb{R}^m \to \mathbb{R}$ as $w(\mathbf{z}) \vcentcolon= \inf_{\mathbf{x} \in C} \, \left\{ f(\mathbf{x}) \colon g(\mathbf{x}) \leq \mathbf{z} \right\}$. Then, $w$ is convex. 
\end{lem}
\begin{proof}
    Let $\alpha \in [0,1]$. Then, for $\mathbf{z}_1, \mathbf{z}_2 \in \mathbb{R}^m$, we have
    \begin{align*}
        w(\alpha\mathbf{z}_1 + (1-\alpha)\mathbf{z}_2) &= \inf_{\mathbf{x} \in C} \, \left\{ f(\mathbf{x}) \colon g(\mathbf{x}) \leq \alpha\mathbf{z}_1 + (1-\alpha)\mathbf{z}_2 \right\} \\
        &\leq \inf_{\mathbf{x} \in C} \, \left\{ f(\mathbf{x}) \colon \mathbf{x} = \alpha\mathbf{x}_1 + (1-\alpha)\mathbf{x}_2, \mathbf{x}_1, \mathbf{x}_2 \in C; \, g(\mathbf{x}_1) \leq\mathbf{z}_1 , g(\mathbf{x}_2) \leq \mathbf{z}_2 \right\}  \\
        &\leq \inf_{\mathbf{x} \in C} \, \left\{ \alpha f(\mathbf{x_1}) + (1-\alpha) f(\mathbf{x}_2) \colon \mathbf{x}_1, \mathbf{x}_2 \in C, \, g(\mathbf{x}_1) \leq\mathbf{z}_1 , g(\mathbf{x}_2) \leq \mathbf{z}_2 \right\} \\
        &= \alpha \inf_{\mathbf{x}_1 \in C} \, \left\{ f(\mathbf{x}) \colon g(\mathbf{x}) \leq \mathbf{z}_1 \right\} + (1-\alpha) \inf_{\mathbf{x} \in C} \, \left\{ f(\mathbf{x}) \colon g(\mathbf{x}) \leq \mathbf{z}_2 \right\} \\
        &= \alpha w(\mathbf{z}_1) + (1-\alpha) w(\mathbf{z}_2). \qedhere
    \end{align*}
\end{proof}

We assume further that $\mu_0 \vcentcolon= \inf_{\mathbf{x} \in \Tilde{C}} \, f(\mathbf{x})$ is finite. 

\begin{thm}[Lagrange Multiplier Rule]
    Suppose there exists $\mathbf{x}_1$ satisfying $g(\mathbf{x}_1) < \mathbf{0}$. Then, $\exists \bm{\Lambda} \geq \mathbf{0}$ such that
    \[
        \mu_0 = \inf_{\mathbf{x} \in C} f(\mathbf{x}) + \bm{\Lambda}^{\top} g(\mathbf{x}).
    \]
    Furthermore, if this infimum is attained at some $\mathbf{x}^* \in \Tilde{C}$, then $\mathbf{x}^*$ minimizes $f$ on $\Tilde{C}$ and
    \[
        \left\langle \bm{\Lambda}, g(\mathbf{x}^*) \right\rangle = 0.
    \]
\end{thm}

Note: The existence of $\mathbf{x}_1$ satisfying $g(\mathbf{x}_1) < 0$ is sometimes called Slater's condition, and $\bm{\Lambda}$ is called the Lagrange multiplier.

\begin{proof}
    Let $B \vcentcolon= \left\{ (\mathbf{z}, y) \in \mathbb{R}^m \times \mathbb{R} \colon \mathbf{z} \leq \mathbf{0}, y \leq \mu_0 \right\}$. Then, $(\mathbf{x}_1, \mu_0 - \epsilon) \in \interior{B}$ for some $\epsilon > 0$ so that $\interior{B} \neq \emptyset$. Moreover, $\interior{B} \cap D = \emptyset$. Thus, there exists a supporting hyperplane
    \[
        \mathcal{H} \vcentcolon= \left\{ (\mathbf{z}, y) \in \mathbb{R}^m \times \mathbb{R} \colon \bm{\Lambda}^{\top}\mathbf{z} + \beta y = \gamma \right\}
    \]
    for suitable $\bm{\Lambda} \in \mathbb{R}^m$, $\beta, \gamma \in \mathbb{R}$ such that $D \subseteq U_{\mathcal{H}}$ and $B \subseteq L_{\mathcal{H}}$. That is, for all $(\mathbf{z}_1, y_1) \in D$, $(\mathbf{z}_2, y_2) \in B$, we have
    \[
        \beta y_1 + \bm{\Lambda}^{\top} \mathbf{z}_1 \geq \gamma \geq \beta y_2 + \bm{\Lambda}^{\top} \mathbf{z}_2.
    \]
    From the definition of $B$, we have $\bm{\Lambda} \geq \mathbf{0}$ and $\beta \geq 0$. Since $(\mathbf{0}, \mu_0) \in B$, substituting this in the above inequality, we get
    \[
        \bm{\Lambda}^{\top}\mathbf{z} + \beta y \geq \beta \mu_0 \quad \forall (\mathbf{z}, y) \in D.
    \]
    We leave it as an exercise to show that $\beta > 0$ (Hint: take $\beta = 0$ and arrive at a contradiction). Without loss of generality, we take $\beta = 1$. Since $(\mathbf{0}, \mu_0) \in \partial B \cap \partial D$, we have
    \begin{align*}
        \mu_0 &= \inf_{(\mathbf{z}, y) \in D} \, y + \bm{\Lambda}^{\top}\mathbf{z} \\
        &\leq \inf_{\mathbf{x} \in C} f(\mathbf{x}) + \bm{\Lambda}^{\top} g(\mathbf{x}) \\
        &\leq \inf_{\mathbf{x} \in \Tilde{C}} f(\mathbf{x}) + \bm{\Lambda}^{\top} g(\mathbf{x}) \\
        &\leq \inf_{\mathbf{x} \in \Tilde{C}} f(\mathbf{x}) \\
        &= \mu_0.
    \end{align*}
    The first and second inequalities follow because the infimum is over successively smaller sets, and the third inequality follows since $\bm{\Lambda}^{\top} g(\mathbf{x}) \leq 0$. Thus, equality must hold at every step, implying both statements of the theorem. 
\end{proof}

\begin{defn}[Saddle Point]
    Given $C \subseteq \mathbb{R}^d$, $D \subseteq \mathbb{R}^m$, and a map $f \colon C \times D \to \mathbb{R}$, a point $(\mathbf{x}^*, \mathbf{y}^*) \in C \times D$ is said to be a \emph{saddle point} of $f$ if
    \[
        f(\mathbf{x}^*, \mathbf{y}) \leq f(\mathbf{x}^*, \mathbf{y}^*) \leq f(\mathbf{x}, \mathbf{y}^*)
    \]
    for all $(\mathbf{x}, \mathbf{y})$ in a relatively open neighbourhood of $(\mathbf{x}^*, \mathbf{y}^*)$ in $C \times D$. 
\end{defn}

Now, suppose that the infimum $\mu_0$ is attained at $\mathbf{x}_0 \in C$. We define the Lagrangian $\mathcal{L} \colon C \times \left\{ \mathbf{y} \in \mathbb{R}^m \colon \mathbf{y} \geq \mathbf{0} \right\} \to \mathbb{R}$ as
\[
    \mathcal{L}(\mathbf{x}, \mathbf{y}) \vcentcolon= f(\mathbf{x}) + \mathbf{y}^{\top} g(\mathbf{x}). 
\]
We then have the following property.

\begin{thm}[Saddle Point Property]
    The Lagrangian $\mathcal{L}(\cdot, \cdot)$ has a saddle point at $(\mathbf{x}_0, \bm{\Lambda})$. That is, 
    \[
        \mathcal{L}(\mathbf{x}_0, \mathbf{y}) \leq \mathcal{L}(\mathbf{x}_0, \bm{\Lambda}) \leq \mathcal{L}(\mathbf{x}, \bm{\Lambda}) \quad \forall \mathbf{x} \in C, \mathbf{0} \leq \mathbf{y} \in \mathbb{R}^m.
    \]
\end{thm}
\begin{proof}
    For any $\mathbf{x} \in C$, by definition of $\mu_0, \mathbf{x}_0$, we have
    \[
        \mu_0 = f(\mathbf{x}_0) + \bm{\Lambda}^{\top} g(\mathbf{x}_0) \leq f(\mathbf{x}) + \bm{\Lambda}^{\top} g(\mathbf{x}).
    \]
    Thus, $\mathcal{L}(\mathbf{x}_0, \bm{\Lambda}) \leq \mathcal{L}(\mathbf{x}, \bm{\Lambda})$ for all $\mathbf{x} \in C$. On the other hand, for all $\mathbf{y} \geq \mathbf{0}$ in $\mathbb{R}^m$, we have
    \begin{align*}
        \mathcal{L}(\mathbf{x}_0, \mathbf{y}) - \mathcal{L}(\mathbf{x}_0, \bm{\Lambda}) &= \mathbf{y}^{\top} g(\mathbf{x}_0) - \bm{\Lambda}^{\top} g(\mathbf{x}_0) \\
        &= \mathbf{y}^{\top} g(\mathbf{x}_0) \\
        &\leq 0
    \end{align*}
    by the Lagrange Multiplier Rule. The result hence follows. 
\end{proof}
\newpage
\section{Lecture 9}

We now turn our attention to \textbf{linear programming}, a specific yet useful instance of convex optimization where all functions involved are linear. A typical linear program in standard form is written as
\begin{equation}
    \label{LP}
    \begin{cases}
        \displaystyle\inf_{\mathbf{x} \in \mathbb{R}^d} \, \langle \mathbf{c}, \mathbf{x} \rangle \quad \text{subject to } \\
        \mathbf{Ax} = \mathbf{b} \text{ and } \\
        \mathbf{x} \geq \mathbf{0},
    \end{cases} \tag{LP}
\end{equation}
where $\mathbf{A} \in \mathbb{R}^{m \times d}$, $\mathbf{b} \in \mathbb{R}^m$, and $\mathbf{c} \in \mathbb{R}^d$. Let $\mathcal{F} \vcentcolon= \left\{ \mathbb{R}^d_+ \colon \mathbf{Ax} = \mathbf{b} \right\}$ denote the feasible set. We let $\mathbf{a}_i$ denote the $i^{\text{th}}$ row of $\mathbf{A}$. The apparently more general problem obtained by replacing the constraint $\mathbf{Ax} = \mathbf{b}$ by the constraint $\mathbf{Ax} \leq \mathbf{b}$ can be reduced to standard form by introducing additional variables such as \textit{slack variables}. It suffices thus to restrict our analysis to the standard form.

The following result is crucial in the simplex method, one of the major algorithms in linear programming. 

\begin{thm}
    If $\mathcal{F}$ has an extreme point and the linear programming problem \eqref{LP} has an optimal solution, then the linear program has an optimal solution which is an extreme point of $\mathcal{F}$.
\end{thm}
\begin{proof}
    Let $S$ denote the set of optimal solutions to \eqref{LP}. It is easy to see that $S$ is closed and convex. Since $\mathcal{F}$ has an extreme point, Theorem 5.6 ensures that $\mathcal{F}$ (and hence $S$) has no lines. By the same theorem, $S$ now has an extreme point $\mathbf{x}^*$. Let $\mathbf{y}, \mathbf{z} \in \mathcal{F}$ be such that $\mathbf{x}^* = \alpha \mathbf{y} + (1-\alpha)\mathbf{z}$ for some $\alpha \in (0,1)$. Then, 
    \[
        \inf_{\mathbf{x} \in \mathcal{F}} \langle \mathbf{c}, \mathbf{x} \rangle = \langle \mathbf{c}, \mathbf{x}^* \rangle = \alpha \left\langle \mathbf{c}, \mathbf{y} \right\rangle + (1-\alpha) \left\langle \mathbf{c}, \mathbf{z} \right\rangle
    \]
    and thus $\langle \mathbf{c}, \mathbf{x}^* \rangle = \langle \mathbf{c}, \mathbf{y} \rangle = \langle \mathbf{c}, \mathbf{z} \rangle$, so that $\mathbf{y}, \mathbf{z} \in S$. Thus, $\mathbf{x}^* = \mathbf{y} = \mathbf{z}$ proving that $\mathbf{x}^*$ is an extreme point of $\mathcal{F}$. 
\end{proof}

Corresponding to the linear program in \eqref{LP}, which we call the \textit{Primal} problem, we have a \textit{Dual} problem given by

\begin{equation}
    \label{dual}
    \begin{cases}
        \displaystyle\sup_{\mathbf{y} \in \mathbb{R}^m} \, \langle \mathbf{y}, \mathbf{b} \rangle \quad \text{subject to } \\
        \mathbf{y}^{\top}\mathbf{A} \leq \mathbf{c}.
    \end{cases} \tag{Dual}
\end{equation}

Define
\begin{align*}
    \alpha &\vcentcolon= \inf_{\mathbf{x} \in \mathcal{F}} \, \left\langle \mathbf{c}, \mathbf{x} \right\rangle \\
    \beta &\vcentcolon= \sup_{\mathbf{y}^{\top}\mathbf{A} \, \leq \, \mathbf{c}} \, \left\langle \mathbf{y}, \mathbf{b} \right\rangle
\end{align*}

\begin{thm}[Weak Duality]
    Let $\mathbf{x}$ be a feasible solution to the Primal and let $\mathbf{y}$ be a feasible solution to the Dual. We have
    \[
        \left\langle \mathbf{y}, \mathbf{b} \right\rangle \leq \left\langle \mathbf{c}, \mathbf{x} \right\rangle.
    \]
    In particular, $\beta \leq \alpha$ always holds. 
\end{thm}
\begin{proof}
    We have $\mathbf{Ax} = \mathbf{b}$ and $\mathbf{y}^{\top}\mathbf{A} \leq \mathbf{c}$. Thus,
    \[
        \left\langle \mathbf{y}, \mathbf{b} \right\rangle = \left\langle \mathbf{y}, \mathbf{Ax} \right\rangle \leq \left\langle \mathbf{y}^{\top}\mathbf{A}, \mathbf{x} \right\rangle \leq \left\langle \mathbf{c}, \mathbf{x} \right\rangle. \qedhere
    \]
\end{proof}

From weak duality, we see that if the primal is unbounded ($\alpha = -\infty$), then the dual is infeasible ($\beta = -\infty$). Similarly, if the dual is unbounded ($\beta = \infty$), then the primal is infeasible ($\alpha = \infty$). 

\begin{thm}[Strong Duality]
    If either of the Primal or Dual is feasible and bounded, then $\beta = \alpha$. 
\end{thm}
\begin{proof}
    If the Primal is feasible and bounded, then the Dual is feasible. Similarly, if the Dual is feasible and bounded, then the Primal is feasible. Now, consider
    \[
        \mathcal{K} = \left\{ \Tilde{\mathbf{A}}\mathbf{x} \colon \mathbf{x} \in \mathbb{R}^d, \mathbf{x} \geq \mathbf{0} \right\}
    \]
    where
    \[
        \Tilde{\mathbf{A}} \vcentcolon= \begin{bmatrix}
            \mathbf{A} \\ \mathbf{c}^{\top} 
        \end{bmatrix} \in \mathbb{R}^{(m+1) \times d}.
    \]
    Then, for any $\epsilon > 0$, we have $\begin{pmatrix}
        \mathbf{b} \\ \alpha - \epsilon
    \end{pmatrix} \notin \mathcal{K}$. Using Theorem 5.3 (Separation Theorem), we can find $\mathbf{y}^{\prime} \in \mathbb{R}^m, \eta \in \mathbb{R}$ such that
    \[
        \left\langle \mathbf{y}^{\prime}, \mathbf{b} \right\rangle + \eta(\alpha - \epsilon) > 0 \geq \left\langle \mathbf{y}^{\prime}, \mathbf{Ax} \right\rangle + \eta \left\langle \mathbf{c}, \mathbf{x} \right\rangle
    \]
    for every $\epsilon > 0$ and $\mathbf{x} \geq \mathbf{0}$. By the feasibility of the primal, there exists $\mathbf{x}$ satisfying $\mathbf{Ax} = \mathbf{b}$, and thus $\eta \neq 0$. At the same time, since $\epsilon$ is arbitrary, we must have $\eta < 0$. Replacing $\mathbf{y}^{\prime}$ by $\mathbf{y} \vcentcolon= -\frac{1}{\eta} \mathbf{y}^{\prime}$, we obtain
    \[
        \left\langle \mathbf{y}, \mathbf{b} \right\rangle > \alpha - \epsilon
    \]
    for every $\epsilon > 0$. Similarly, for every $\mathbf{x} \geq \mathbf{0}$, we have
    \[
        \left\langle \mathbf{y}, \mathbf{Ax} \right\rangle \leq \left\langle \mathbf{c}, \mathbf{x} \right\rangle. 
    \]
    Since $\mathbf{x} \geq \mathbf{0}$ is arbitrary, we must have $\mathbf{y}^{\top}\mathbf{A} \leq \mathbf{c}^{\top}$. Thus, $\mathbf{y}$ is feasible for Dual and hence $\beta \leq \alpha - \epsilon$ for every $\epsilon > 0$ so that $\beta \geq \alpha$. Coupled with Weak Duality, this gives us $\beta = \alpha$. 
\end{proof} 

One standard application of strong duality is in Game Theory, where it is most commonly used to prove \href{https://en.wikipedia.org/wiki/Minimax_theorem}{von Neumann's Minimax Theorem} for two-player zero-sum games. 

We state two further results regarding the solvability of linear systems without proof. 

\begin{thm}[Gauss]
    Let $\mathbf{A} \in \mathbb{R}^{m \times d}$ and $\mathbf{b} \in \mathbb{R}^m$. Then exactly one of the following holds. 

    \begin{enumerate}
        \item There exists an $\mathbf{x} \in \mathbb{R}^d$ such that $\mathbf{Ax} = \mathbf{b}$.
        \item There exists a $\mathbf{y} \in \mathbb{R}^m$ such that $\mathbf{y}^{\top}\mathbf{A} = \mathbf{0}$, $\mathbf{y}^{\top}\mathbf{b} \neq 0$.
    \end{enumerate}
\end{thm}

\begin{lem}[Farkas' Lemma]
    Let $\mathbf{A} \in \mathbb{R}^{m \times d}$ and $\mathbf{b} \in \mathbb{R}^m$. Then exactly one of the following holds. 

    \begin{enumerate}
        \item There exists an $\mathbf{x} \in \mathbb{R}^d$ such that $\mathbf{Ax} = \mathbf{b}$ with $\mathbf{x} \geq \mathbf{0}$.
        \item There exists a $\mathbf{y} \in \mathbb{R}^m$ such that $\mathbf{y}^{\top}\mathbf{A} \leq \mathbf{0}$, $\mathbf{y}^{\top}\mathbf{b} > 0$.
    \end{enumerate}
\end{lem}

Further, these two results are equivalent. 
\newpage
\section{Lecture 10}

Recall that optimal solutions to linear programs are extreme points of the feasible set. One way to optimize is then to simply search over extreme points. This is exactly what the \textbf{simplex algorithm} does, which we describe during the course of this lecture. We start at some initial extreme point and then check ``neighbors'' of that extreme point, by replacing some constraints with others. As this is a convex optimization problem, any local minimum is also a global minimum and we are done. 

We recall the setting of the standard form LP:

\[
    \begin{cases}
        \displaystyle\inf_{\mathbf{x} \in \mathbb{R}^d} \, \langle \mathbf{c}, \mathbf{x} \rangle \quad \text{subject to } \\
        \mathbf{Ax} = \mathbf{b} \text{ and } \\
        \mathbf{x} \geq \mathbf{0},
    \end{cases}
\]
where $\mathbf{A} \in \mathbb{R}^{m \times d}$, $\mathbf{b} \in \mathbb{R}^m$, and $\mathbf{c} \in \mathbb{R}^d$. We can always take $m \leq d$ because if $m > d$, some of the constraints would be a linear combination of others and we can remove them. By the same logic, we may assume $\rank(\mathbf{A}) = m$. We then pick a set of indices $B \subseteq \{1, \ldots, d\}$ that correspond to $m$ linearly independent rows of $\mathbf{A}$. Now, we may write
\[
    \mathbf{A} = \begin{bmatrix}
        \mathbf{A}_B & \mathbf{A}_N 
    \end{bmatrix} \quad \text{and} \quad \mathbf{x} = \begin{bmatrix}
        \mathbf{x}_B & \mathbf{x}_N
    \end{bmatrix}
\]
where $N \vcentcolon= \{1, \ldots, n\} \setminus B$. We may then take $\mathbf{x}_N = \mathbf{0}$ and compute $\mathbf{x}_B = \mathbf{A}_B^{-1} \mathbf{b}$. This solution is called a \textit{basic feasible solution} (BFS). The simplex algorithm works as shown on the next page. 

\begin{algorithm}[!ht]
  \caption{Simplex Algorithm}\label{simplex}
  \begin{algorithmic}[1]
    \Function{Simplex}{$\mathbf{A} \in \mathbb{R}^{m \times d}, \mathbf{b} \in \mathbb{R}^m, \mathbf{c} \in \mathbb{R}^d$}

    \While{True}
    
    \State Compute BFS $(\mathbf{x}_B, \mathbf{x}_N)$ with $\mathbf{x}_N = \mathbf{0}$, $\mathbf{x}_B = \mathbf{A}_B^{-1}\mathbf{b}$
    
    \State $\overline{\mathbf{c}}^{\top} \gets \mathbf{c}_N^{\top} - \mathbf{c}_B^{\top} \mathbf{A}_B^{-1} \mathbf{A}_N$

    \If{$\overline{\mathbf{c}} \geq \mathbf{0}$}
        \State STOP and return $(\mathbf{x}_B, \mathbf{x}_N)$
    \EndIf

    \State Select $j \in N$ such that $\overline{\mathbf{c}}_j < 0$

    \State $\mathbf{d}_j \gets \mathbf{A}_B^{-1} \mathbf{a}_j$

    \If{$\mathbf{d}_j \leq \mathbf{0}$}
        \State STOP and return (LP unbounded)
    \EndIf

    \State $k \gets \displaystyle\argmin_{i \in B \colon d_{ji} > 0} \, \overline{b}_i / d_{ji}$

    \State $B \gets (B \setminus \{k\}) \cup \{j\}$, $N \gets (N \setminus \{j\} \cup \{k\}$

    \EndWhile
    \EndFunction
  \end{algorithmic}
\end{algorithm}

\newpage

Even though this algorithm terminates in a finite number of steps, it can have an exponential complexity ($d^m$) in the worst case. In practice, however, it often performs better than the polynomial time algorithm.

Linear Programming often finds applications in everyday life. Some examples include the \href{https://towardsdatascience.com/operations-research-in-r-transportation-problem-1df59961b2ad}{Transportation problem}, the \href{https://optimization.mccormick.northwestern.edu/index.php/Network_flow_problem#:~:text=Network%20Flow%20Optimization%20problems%20form,viewed%20as%20minimizing%20transportation%20problems.}{Network Flow Problem}, and the \href{https://en.wikipedia.org/wiki/Chebyshev_center}{Chebyshev Center problem}.
\newpage
\section{Lecture 11}

We shift our attention to Non-Linear Programming, a problem considerably harder than Linear Programming. Given an objective function $f \colon \mathbb{R}^d \to \mathbb{R}$, we wish to find a local minimum $\mathbf{x}^* \in \mathbb{R}^d$ of $f$. One method to do so is called the \textbf{line search}, which is an iterative method. We start with some initial guess $\mathbf{x}_0$ and use the current guess to compute the next guess. This gives us a sequence of estimates
\[
    \mathbf{x}_0 \to \mathbf{x}_1 \to \mathbf{x}_2 \to \cdots \to \mathbf{x}_n \to \mathbf{x}_{n+1} \to \cdots
\]
which we hope eventually converges to $\mathbf{x}^*$. A general line search iteration can be written as
\[
    \mathbf{x}_{n+1} = \mathbf{x}_n + \alpha_n \cdot \mathbf{d}_n.
\]
Two questions then naturally arise.
\begin{enumerate}
    \item What direction should we move in? (That is, how do we pick $\mathbf{d}_n$?)
    \item How far should we move in this direction? (That is, how do we pick $\alpha_n$?)
\end{enumerate}

A naïve way to proceed is to use exact line minimization. That is, at every step we solve the following scalar optimization problem:
\[
    \alpha_n \vcentcolon= \argmin_{\alpha} f(\mathbf{x}_n + \alpha \cdot \mathbf{d}_n)
\]
where the search takes place over a suitable subset of $\mathbb{R}^+$. This problem, however, may not be easy or fast to solve in the general non-convex case. Moreover, the above minimization represents a \textit{greedy} heuristic, and it is not clear if this is really the best thing to do.

The more common approach is to restrict the line search to an interval $\mathcal{I} = [0,r]$ for a suitable $r < \infty$. Further, we relax the exact minimization objective and instead search for points that give us an adequate decrease in the objective function. A classical scheme to do so relies on a `search and compare' idea wherein we successively generate subintervals of $\mathcal{I}$ recursively by subdividing the previous interval at each step. One example is the simple binary search method which bisects the interval at each step. Some more sophisticated methods include the Fibonacci search where intervals are subdivided according to the ratio of successive Fibonacci numbers and the golden section method where the ratio is fixed at the golden ratio (which, as it turns out, is the limiting ratio of two successive Fibonacci numbers). These latter two algorithms have some theoretically proven advantages over the simple binary search method.

Even in such cases, however, there is no guarantee of convergence and the iterates may keep oscillating. To fix this, we intuitively require that the direction $\mathbf{d}_n$ behaves ``gradient-like''. More concretely, we hope that
\[
    \left\langle \nabla f(\mathbf{x}_n), \mathbf{d}_n \right\rangle < 0
\]
so that by the Taylor formula, we have
\begin{align*}
    f(\mathbf{x}_{n+1}) &= f(\mathbf{x}_n) + \alpha_n \cdot \left\langle \nabla f(\mathbf{x}_n), \mathbf{d}_n \right\rangle + o(\alpha_n) \\
    &< f(\mathbf{x}_n) + o(\alpha_n).
\end{align*}

Note that the $\left\langle \nabla f(\mathbf{x}_n), \mathbf{d}_n \right\rangle$ may be quite close to $0$ even for non-negligible values of $\nabla f(\mathbf{x}_n), \mathbf{d}_n$ unless the angle between the two is bounded away from $\pm \frac{\pi}{2}$. Thus, what we really require is
\[
    \left\langle \nabla f(\mathbf{x}_n), \mathbf{d}_n \right\rangle < -\delta_n
\]
for a judiciously chosen $\delta_n > 0$.

A popular variant of this scheme is the \textbf{Armijo rule}, which we describe next. We first impose the following two conditions:
\begin{align*}
    \sup_n \, \norm{\mathbf{d}_n} &< \infty, \\
    \sup_n \left\langle \nabla f(\mathbf{x}_n), \mathbf{d}_n \right\rangle &< 0.
\end{align*}

The Armijo rule is as follows. Let $\mathcal{I} = [0,r]$ be fixed as before. Fix $0 < \sigma, \beta < 1$. Set $\alpha_n \vcentcolon= \beta^{m(n)} r$ where
\[
    m(n) \vcentcolon= \left\{ m \geq 0 \colon f(\mathbf{x}_n) - f(\mathbf{x}_n + \beta^m r \mathbf{d}_n) \geq -\sigma \beta^m r \left\langle \nabla f(\mathbf{x}_n), \mathbf{d}_n \right\rangle  \right\}
\]

We then have the following result.

\begin{thm}
    If $\mathbf{x}_n \to \mathbf{x}^*$ along a subsequence, then $\nabla f(\mathbf{x}^*) = \mathbf{0}$.
\end{thm}
\begin{proof}
    Let $\mathbf{x}_{n_k} \to \mathbf{x}^*$ along a subsequence and suppose that $\nabla f(\mathbf{x}^*) \neq \mathbf{0}$. We have
    \[
        f(\mathbf{x}_{n_k} - f(\mathbf{x}_{n_k + 1}) \to 0.
    \]
    However, 
    \begin{align*}
        f(\mathbf{x}_{n_k}) - f(\mathbf{x}_{n_k + 1}) &\geq -\alpha_{n_k} \left\langle \nabla f(\mathbf{x}_n), \mathbf{d}_n \right\rangle \\
        &\geq 0 \quad (\text{by the hypothesis that } \left\langle \nabla f(\mathbf{x}_n), \mathbf{d}_n \right\rangle < 0).
    \end{align*}
    Thus, we must have
    \[
        \alpha_{n_k} \left\langle \nabla f(\mathbf{x}_n), \mathbf{d}_n \right\rangle \to 0 \implies \alpha_{n_k} \to 0
    \]
    where the latter follows since $\sup_n \left\langle \nabla f(\mathbf{x}_n), \mathbf{d}_n \right\rangle < 0$. Now,
    \[
        f(\mathbf{x}_{n_k}) - f\left( \mathbf{x}_{n_k} + \frac{\alpha_{n_k}}{\beta}\mathbf{d}_{n_k} \right) < -\sigma  \frac{\alpha_{n_k}}{\beta} \left\langle \nabla f(\mathbf{x}_n), \mathbf{d}_n \right\rangle.
    \]
    Next, define
    \begin{align*}
        \mathbf{p}_k &\vcentcolon= \frac{\mathbf{d}_{n_k}}{\norm{\mathbf{d}_{n_k}}} \\  b_k &\vcentcolon= \frac{\alpha_{n_k} \norm{\mathbf{d}_{n_k}}}{\beta} \\
        \mathbf{x}^{\prime}_k &\vcentcolon= \mathbf{x}_{n_k} \\
        \mathbf{d}^{\prime}_k &\vcentcolon= \mathbf{d}_{n_k}.
    \end{align*}
    By the Bolzano-Weierstrass Theorem, $\mathbf{p}_k \to \mathbf{p}^*$ along a subsequence (since these are unit vectors and bounded), which we shall denote as $\{\mathbf{p}_k\}$ again. Rewriting with this new notation, we have
    \[
        \frac{f(\mathbf{x}^{\prime}_k) - f\left( \mathbf{x}^{\prime}_k + b_k \mathbf{p}_k \right)}{b_k} < -\sigma \cdot \left\langle \nabla f(\mathbf{x}^{\prime}_k), \mathbf{p}_k \right\rangle
    \]
    Thus, for some $c_k \in [0,b_k]$, we have
    \[
        -\left\langle \nabla f(\mathbf{x}^{\prime}_k + c_k \mathbf{p}_k), \mathbf{p}_k \right\rangle < -\sigma \left\langle \nabla f(\mathbf{x}^{\prime}_k), \mathbf{p}_k \right\rangle
    \]
    Letting $k \uparrow \infty$, we have
    \[
        -\left\langle \nabla f(\mathbf{x}^*), \mathbf{p}^* \right\rangle \leq -\sigma \left\langle \nabla f(\mathbf{x}^*), \mathbf{p}^* \right\rangle \implies \left\langle \nabla f(\mathbf{x}^*), \mathbf{p}^* \right\rangle \geq 0.
    \]
    However, we also have
    \[
        \left\langle \nabla f(\mathbf{x}^{\prime}_k), \mathbf{p}_k \right\rangle = \frac{\left\langle \nabla f(\mathbf{x}^{\prime}_k), \mathbf{d}_k \right\rangle}{\norm{\mathbf{d}_k}}
    \]
    Again, letting $k \uparrow \infty$, 
    \[
        \left\langle \nabla f(\mathbf{x}^*), \mathbf{p}^* \right\rangle = \lim_{k \uparrow \infty} \frac{\left\langle \nabla f(\mathbf{x}^{\prime}_k), \mathbf{d}_k \right\rangle}{\norm{\mathbf{d}_k}} < 0
    \]
    which is a contradiction. 
\end{proof}

At every step $\mathbf{d}_k$ in terms of the current iterate $\mathbf{x}_k$ which allows us to write $\mathbf{d}_k = f(\mathbf{x}_k)$ for some appropriate $f \colon \mathbb{R}^d \to \mathbb{R}^d$. It helps to view the algorithm as a discretization (Euler scheme) of the ODE
\[
    \dot{\mathbf{x}}(t) = f(\mathbf{x}(t)).
\]
As a result, we need
\[
    \sum_{n=1}^{\infty} \alpha_n = \infty
\]
to make sure the entire time axis is covered. 

\subsection*{Gradient Descent}

A popular variant of the iterative algorithm is \textbf{Gradient Descent} wherein we have the following update rule:
\[
    \mathbf{x}_{k+1} = \mathbf{x}_k - \alpha_k \cdot \nabla f(\mathbf{x}_k).
\]
As a result, we have
\[
    f(\mathbf{x}_{k+1}) = f(\mathbf{x}_k) - \alpha_k \norm{\nabla f(\mathbf{x}_k)}^2 + o(\alpha_k).
\]
This guarantees convergence to points where $\nabla f(\mathbf{x}) = \mathbf{0}$, which we call \textit{critical points}. However, such points need not necessarily be local minima (for example, they can be local maxima too). However, we don't usually run into this problem due to the reasons described below. 

\begin{defn}[Isolated Critical Points]
    A critical point $\mathbf{x}^*$ of $f$ is said to be \emph{isolated} if there is an open neighborhood $\mathcal{O}$ of $\mathbf{x}^*$ that contains no other critical points of $f$.     
\end{defn}

For most ``nice'' functions, critical points are isolated and so we are more or less guaranteed to converge to these points. There may be cases where we start exactly at a local maximum in which case our iterate remains unchanged. However, this is highly unlikely and thus convergence to a local minimum is almost guaranteed. 
\end{document}