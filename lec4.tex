\section{Lecture 4}

\begin{thm}[Envelope Theorem / Danskin's Theorem]
    Suppose $C \subseteq \mathbb{R}^d$ is open and $D \subseteq \mathbb{R}^m$ is closed and bounded. Suppose that $f \colon C \times D \to \mathbb{R}$ is continuous and its partial gradient with respect to $\mathbf{x} \in C$, denoted as
    \[
        \nabla^{\mathbf{x}} f(\mathbf{x}, \mathbf{y}) \vcentcolon= \begin{bmatrix}
            \frac{\partial f}{\partial x_1}(\mathbf{x}) & \cdots & \frac{\partial f}{\partial x_d}(\mathbf{x})
        \end{bmatrix} \in \mathbb{R}^d
    \]
    is continuous. Let $g(\mathbf{x}) \vcentcolon= \max_{\mathbf{y} \in D} \, f(\mathbf{x}, \mathbf{y})$, where the maximum is attained on a non-empty closed and bounded set $M(\mathbf{x}) \subseteq D$. Then, $g \colon \mathbb{R}^d \to \mathbb{R}$ has a directional derivative in every direction, given by
    \[
        g^{\prime}(\mathbf{x} ; \mathbf{n}) = \max_{\mathbf{y} \in M(\mathbf{x})} \, \left\langle \nabla^{\mathbf{x}} f(\mathbf{x}, \mathbf{y}), \mathbf{n} \right\rangle
    \]
    for every unit vector $\mathbf{n} \in \mathbb{R}^d$. 
\end{thm}

\begin{defn}[Convex Set]
    A set $C \subseteq \mathbb{R}^d$ is said to be \emph{convex} if for all $\mathbf{x}, \mathbf{y} \in C$, we have
    \[
        \alpha\mathbf{x} + (1-\alpha)\mathbf{y} \in C \quad \forall \alpha \in [0,1].
    \]
\end{defn}

\begin{prop}
    A set $C \subseteq \mathbb{R}^d$ is convex iff $\forall \mathbf{x}_1, \ldots \mathbf{x}_n \in C$, $n > 0$ and $\alpha_i > 0$ with $\sum_i \alpha_i = 1$, we have
    \[
        \sum_{i=1}^n \alpha_i \mathbf{x}_i \in C.
    \]
\end{prop}
\begin{proof}
    $(\impliedby)$ is clear by taking $n = 2$. For $(\implies)$, we already know that the result holds for $n = 2$, by definition. Suppose the statement holds for some $n \geq 2$. Then, 
    \[
        \sum_{i=1}^{n+1} \alpha_i \mathbf{x}_i = \alpha_1 \mathbf{x}_1 + (1-\alpha_1) \cdot \underbrace{\sum_{i=2}^{n+1} \frac{\alpha_i}{1-\alpha_1} \mathbf{x}_i}_{\in C} \in C,
    \]
    where the latter point is in $C$ by the induction hypothesis. The result then follows from induction.
\end{proof}
Note that if $C$ is closed, it suffices to check that $\mathbf{x}, \mathbf{y} \in C \implies \frac{\mathbf{x} + \mathbf{y}}{2} \in C$. 

We list out some properties of convex sets below. 

\begin{enumerate}
    \item Convex sets are connected.
    \item Intersection of an arbitrary family of convex sets is convex.
    \item Union of two convex sets need not be convex.
    \item Interior and closure of convex sets are convex.
    \item Image of a convex set under an affine map is convex.
\end{enumerate}

\begin{defn}[Convex Hull]
    Let $A \subseteq \mathbb{R}^d$. The \emph{convex hull} of $A$, denoted $\co(A)$, is the smallest convex set containing $A$, or equivalently, the intersection of all convex sets containing $A$, or equivalently, the set of convex combinations of all points in $A$. 
\end{defn}

\begin{defn}[Closed Convex Hull]
    Let $A \subseteq \mathbb{R}^d$. The \emph{closed convex hull} of $A$, denoted $\cco(A)$, is the smallest closed convex set containing $A$, or equivalently, the intersection of all closed convex sets containing $A$.
\end{defn}

\begin{thm}
    Let $C \subseteq \mathbb{R}^d$ be closed and convex and let $\mathbf{x} \notin C$. Then, there exists a unique $\mathbf{x}^* \in C$ such that
    \[
        \norm{\mathbf{x} - \mathbf{x}^*} = \min_{\mathbf{y} \in C} \, \norm{\mathbf{x} - \mathbf{y}}.
    \]
\end{thm}

\begin{proof}
    Note that $\mathbf{y} \mapsto \norm{\mathbf{x} - \mathbf{y}}$ is a continuous map. By triangle inequality, $\norm{\mathbf{x} - \mathbf{y}} \geq \norm{\mathbf{y}} - \norm{\mathbf{x}}$, and thus
    \[
        \lim_{\norm{\mathbf{y}} \uparrow \infty} \norm{\mathbf{x} - \mathbf{y}} = \infty.
    \]
    The existence of a minimizer $\mathbf{x}^*$ now follows from Corollary 2.1 of the Weierstrass Theorem. Suppose $\hat{\mathbf{x}} \neq \mathbf{x}^*$ is another minimizer. Then, the triangle formed by $(\mathbf{x}, \mathbf{x}^*, \hat{\mathbf{x}})$ is an isosceles triangle with line segment $(\mathbf{x}^*, \hat{\mathbf{x}})$ as its base. Moreover, this triangle lies completely in $C$ by convexity. By elementary geometry, it is easy to see that the midpoint $\frac{\mathbf{x}^* + \hat{\mathbf{x}}}{2}$ is at a strictly smaller distance from $\mathbf{x}$ than $\mathbf{x}^*, \hat{\mathbf{x}}$, a contradiction. Thus, $\mathbf{x}^*$ is the unique minimizer of $\norm{\mathbf{y} - \mathbf{x}}$ over $C$. 
\end{proof}

The above $\mathbf{x}^*$ is called the \emph{projection} of $\mathbf{x}$ onto $C$. This theorem immediately adapts to a more general setting, as follows.

\begin{thm}
    Let $C,D \subseteq \mathbb{R}^d$ be disjoint closed convex sets with $C$ bounded. Then, there exist $\mathbf{x}^* \in C$, $\mathbf{y}^* \in D$ such that
    \[
        0 < \norm{\mathbf{x}^* - \mathbf{y}^*} = \min_{\mathbf{x} \in C , \mathbf{y} \in D} \, \norm{\mathbf{x} - \mathbf{y}}.
    \]
\end{thm}
\begin{proof}
    Consider the map $(\mathbf{x}, \mathbf{y}) \in C \times D \mapsto \norm{\mathbf{x} - \mathbf{y}}$, which is clearly continuous. If $D$ is also bounded, the claim is immediate from Weierstrass Theorem (Theorem 2.3). If not, we have
    \[
        \norm{(\mathbf{x}, \mathbf{y})} \uparrow \infty \iff \norm{\mathbf{y}} \uparrow \infty,
    \]
    since $C$ is bounded. Thus, $\norm{(\mathbf{x}, \mathbf{y})} \uparrow \infty \implies \norm{\mathbf{x} - \mathbf{y}} \uparrow \infty$. The existence of a minimizing pair $(\mathbf{x}^*, \mathbf{y}^*)$ again follows from Corollary 2.1 of the Weierstrass Theorem. Moreover, since $C,D$ are disjoint, $\norm{\mathbf{x}^* - \mathbf{y}^*} > 0$.
\end{proof}
Note that no uniqueness can be claimed. (Hint: imagine two disjoint rectangles in $\mathbb{R}^2$ with sides parallel to each other). Next, we characterize the projection of $\mathbf{x}$ onto $C$.

\begin{thm}
    Let $C \subseteq \mathbb{R}^d$ be closed convex and let $\mathbf{x} \notin C$. Then,
    \[
        \mathbf{x}^* = \argmin_{\mathbf{y} \in C} \norm{\mathbf{x} - \mathbf{y}}
    \]
    if and only if
    \[
        \left\langle \mathbf{y} - \mathbf{x}^*, \mathbf{x} - \mathbf{x}^* \right\rangle \leq 0.
    \]
\end{thm}
\begin{proof}
    The proof is left as an exercise. 
\end{proof}